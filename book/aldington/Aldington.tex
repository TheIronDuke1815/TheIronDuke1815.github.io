% Options for packages loaded elsewhere
\PassOptionsToPackage{unicode}{hyperref}
\PassOptionsToPackage{hyphens}{url}
%
\documentclass[
  oneside,
  12pt,
  titlepage]{book}
\usepackage{lmodern}
\usepackage{amssymb,amsmath}
\usepackage{ifxetex,ifluatex}
\ifnum 0\ifxetex 1\fi\ifluatex 1\fi=0 % if pdftex
  \usepackage[T1]{fontenc}
  \usepackage[utf8]{inputenc}
  \usepackage{textcomp} % provide euro and other symbols
\else % if luatex or xetex
  \usepackage{unicode-math}
  \defaultfontfeatures{Scale=MatchLowercase}
  \defaultfontfeatures[\rmfamily]{Ligatures=TeX,Scale=1}
\fi
% Use upquote if available, for straight quotes in verbatim environments
\IfFileExists{upquote.sty}{\usepackage{upquote}}{}
\IfFileExists{microtype.sty}{% use microtype if available
  \usepackage[]{microtype}
  \UseMicrotypeSet[protrusion]{basicmath} % disable protrusion for tt fonts
}{}
\makeatletter
\@ifundefined{KOMAClassName}{% if non-KOMA class
  \IfFileExists{parskip.sty}{%
    \usepackage{parskip}
  }{% else
    \setlength{\parindent}{0pt}
    \setlength{\parskip}{6pt plus 2pt minus 1pt}}
}{% if KOMA class
  \KOMAoptions{parskip=half}}
\makeatother
\usepackage{xcolor}
\IfFileExists{xurl.sty}{\usepackage{xurl}}{} % add URL line breaks if available
\IfFileExists{bookmark.sty}{\usepackage{bookmark}}{\usepackage{hyperref}}
\hypersetup{
  pdftitle={Жизнь Веллингтона. Герцог},
  pdfauthor={Ричард Олдингтон},
  hidelinks,
  pdfcreator={LaTeX via pandoc}}
\urlstyle{same} % disable monospaced font for URLs
\usepackage{longtable,booktabs}
% Correct order of tables after \paragraph or \subparagraph
\usepackage{etoolbox}
\makeatletter
\patchcmd\longtable{\par}{\if@noskipsec\mbox{}\fi\par}{}{}
\makeatother
% Allow footnotes in longtable head/foot
\IfFileExists{footnotehyper.sty}{\usepackage{footnotehyper}}{\usepackage{footnote}}
\makesavenoteenv{longtable}
\usepackage{graphicx,grffile}
\makeatletter
\def\maxwidth{\ifdim\Gin@nat@width>\linewidth\linewidth\else\Gin@nat@width\fi}
\def\maxheight{\ifdim\Gin@nat@height>\textheight\textheight\else\Gin@nat@height\fi}
\makeatother
% Scale images if necessary, so that they will not overflow the page
% margins by default, and it is still possible to overwrite the defaults
% using explicit options in \includegraphics[width, height, ...]{}
\setkeys{Gin}{width=\maxwidth,height=\maxheight,keepaspectratio}
% Set default figure placement to htbp
\makeatletter
\def\fps@figure{htbp}
\makeatother
\setlength{\emergencystretch}{3em} % prevent overfull lines
\providecommand{\tightlist}{%
  \setlength{\itemsep}{0pt}\setlength{\parskip}{0pt}}
\setcounter{secnumdepth}{5}
\usepackage[a4paper, margin=22mm]{geometry}
\setlength{\headheight}{15.1pt}
\setlength {\parindent}{2em} 
\usepackage{parskip}
\renewcommand{\baselinestretch}{1.3}
\usepackage{titlesec}
\usepackage{fancyhdr}
\renewcommand{\thechapter}{\Roman{chapter}}
\flushbottom
\pagestyle{fancy} 
\fancyhf{} 
\fancyfoot[CO]{\thepage} 
\fancyhead [RO]{\leftmark} 
\titleformat{\chapter}[display]{\center\Large}{Глава \thechapter}{0.1em}{}
\titlespacing*{\chapter}{0pt}{-50pt}{25pt}
\usepackage{indentfirst}
\usepackage[T2A]{fontenc}
\usepackage[utf8]{inputenc}
\usepackage[russianb]{babel}

\title{Жизнь Веллингтона. Герцог}
\author{Ричард Олдингтон}
\date{}

\begin{document}
\maketitle

{
\setcounter{tocdepth}{1}
\tableofcontents
}
\hypertarget{ux432ux43cux435ux441ux442ux43e-ux432ux432ux435ux434ux435ux43dux438ux44f}{%
\chapter{Вместо введения}\label{ux432ux43cux435ux441ux442ux43e-ux432ux432ux435ux434ux435ux43dux438ux44f}}

``Герцог\ldots{}'', ``Старый Герцог'', ``Во времена старого Герцога''.

Такими фразами пользовались в моем присутствии старые люди; и, слушая их с бесхитростной невнимательностью ребенка, я достаточно хорошо понимал, что означают эти слова. В Англии 1900 года слово ``Герцог'' могло относиться лишь к одной персоне. У меня же были особые основания побольше знать о ``Герцоге'', поскольку свыше трех лет я жил возле старого городка Уолмер, а школа моя выходила окнами на парк замка Уолмер, официальной резиденции обладателя этого почетного титула - лорда-хранителя Пяти Портов\footnote{Пять Портов - военная ассоциация городов, расположенных на юго-востоке Англии и пользовавшихся особыми привилегиями за участие в обороне страны. Первоначально к их числу относились Гастингс, Ромни, Хис, Дувр и Сандвич.}. Новым ученикам школы всегда говорили: ``Знаешь, здесь жил герцог Веллингтон''. С той же степенью достоверности можно было бы сказать, что он умер здесь. По-крокодильи припадая к земле, мы огибали замок и прилегающие к нему территории, при этом с завистью обсуждая предполагаемую роскошь, праздность и великолепие, в которые погружен невидимый (и вне сомнения, отсутствующий) лорд-хранитель.

Из таких впечатлений формируются незабываемые воспоминания детства, и до сего дня в слове ``Герцог'' я вижу отпечаток величия, свидетельством которого стало мнение довольно неожиданного ценителя человеческого великолепия, а именно, служанки.

Путь к этому был довольно сложен. Покойный маркиз Керзон оф Кеддлстон\footnote{Джордж Керзон, 1-й маркиз Керзон (1859-1925) - вице-король Индии, министр иностранных дел Великобритании, лидер палаты лордов, лорд-председатель Совета.}, посетив Калькутту, обнаружил кузена своего старинного рода в помпезном правительственном дворце, сооруженном Ричардом Уэлсли\footnote{Ричард Уэлсли Колли, 1-й маркиз Уэлсли (1760-1842) - генерал-губернатор Индии (1798-1805), министр иностранных дел Великобритании (1809-12), лорд-лейтенант Ирландии в 1821-29 и 1833-34 годах.}, генерал-губернатором Индии и старшим братом Веллингтона. Керзон немедленно обнаружил желание ``перебраться из Кеддлстона Английского в Кеддлстон Индийский'', и он исполнил свое намерение с необычайной помпой и настойчивостью. Осуществив свою историческую прихоть и походив, так сказать, в старых сапогах лорда Уэлсли, Керзон, по возвращении в Англию, возжелал уже спать и в постели, оставленной герцогом Веллингтоном.

Как можно было добиться этого? При всем высоком мнении Керзона о себе, он не мог претендовать на роль главнокомандующего и не имел других оснований добиваться передачи ему Эпсли-Хаус\footnote{Эпсли-Хаус - резиденция герцога Веллингтона в Лондоне в юго-восточном углу Гайд-Парка на площади Гайд-Парк Корнер.} в качестве резиденции; однако, в итоге, он разрешил эту колоссальную проблему, добившись, чтобы его назначили лордом-хранителем (что давало ему право обитать в замке Уолмер), и отпраздновал назначение величественным военным шествием по улицам Дувра. Я прекрасно помню процессию, потому что следил за ней из конторы собственного отца, находившейся в этом городе. День выдался жарким, и солнечные лучи проливались прямо на Замковую улицу, вдоль которой выстроились солдаты из полков гарнизона, в теплых красных мундирах, тяжелых шипастых шлемах, при безупречно начищенных трубочной глиной поясах и ранцах, не говоря уже о винтовках, штыках и прочем вооружении. Они заняли свои места достаточно рано и ждали, ждали лорда Керзона, чрезвычайно запаздывавшего со своей процессией, как подобает бывшему деспоту из страны великолепного востока. Время от времени с улицы доносился многозначительный стук ставней, иногда мимо проносили носилки с бледным и обеспамятевшим солдатом или с солдатом, багровым от солнечного удара, павшим жертвой армейской строгости в вопросах одежды и честолюбия лорда Керзона. Но солдатам размышлять о причинах не полагалось.

Наконец и сам лорд Керзон проследовал сквозь подвергшийся децимации\footnote{Децимация - наказание каждого десятого.} строй к концу длинной процессии, непосредственно за пышно наряженным кавалерийским отрядом, поднимавшим своими литаврами и трубами такой грохот, от которого дребезжали окна. Предательница память нашептывает мне, что его лордство восседал на слоне, чего, впрочем, не могло быть. Во-первых, слона пришлось бы позаимствовать в зоопарке или цирке, во-вторых, подобное проявление деспотизма было бы встречено в штыки внушительной и ``радикально и нонконформистским образом настроенной'' частью населения Дувра. По всей видимости, к моим воспоминаниям примешиваются чьи-либо рассказы о фантастическом дурбаре\footnote{Дурбар - от перс. durbar - царский двор - совет знати про монархе или торжественный прием.}, учиненном лордом Керзоном в Дели.

Маленькие дети не склонны размышлять о социальных и моральных последствиях дурацкого и дорогостоящего представления, и, вынужден признаться, шум, мундиры и блеск штыков произвели на меня приятное впечатление. Лорд Керзон, решил я, бесспорно, является истинно великим человеком.

Керзоны заняли свою резиденцию в замке Уолмер, и через несколько дней о них забыли, хотя его лордство, безусловно, блаженствовал, расхаживая по комнатам Веллингтона и, возможно, почивая в его постели, - правда, скорее всего, он не спал на аскетически жестком и узком ложе Герцога. Однако у Керзонов возникли сложности с домашней прислугой, и одна из отвергнутых ими молодых женщин устроилась у нас служанкой.

При всем моем природном тяготении к низкому обществу, я скоро подружился с этой молодой особой и нередко сиживал вместе с ней на кухне или торчал в судомойке, помогая чистить ножи в машине, которую мы, между прочим, потчевали ``Веллингтоновским порошком для ножей Эмери''. Любимой темой ее рассказов были те сложности и оскорбления, которые ей пришлось вытерпеть в замке. Получалось, что за самой первой трапезой, когда она сидела в одиночестве за столом, явился старший дворецкий. Она бестрепетно продолжала есть, и, подойдя к ней, он прошипел: ``Встань, когда с тобой разговаривают старшие''. Тогда ей пришлось встать. Естественным образом переходя от старшего дворецкого к личности лорда Керзона, она сердито брюзжала: ``Вота еще што! И хто он такой, хотела бы я знать, штобы важничать, будто он и есть сам великий Герцог, благослови его Господь''.

Я погрузился в задумчивость. Вот я, смиренный и иногда распекаемый приятель служанки, которой случалось стоять перед своим господином, великим повелителем шествий и барабанов. И, тем не менее, даже ему, великому Керзону, пришлось бы встать перед герцогом Веллингтоном. Насколько же важной персоной был этот Герцог!

Таким образом, смутная память о Веллингтоне более чем через половину столетия после его смерти еще обитала в сердцах людей, населявших место, где он жил и умер, и бессознательно передавалась ими собственным детям.

Немного позже мне открылся Веллингтон как Национальный Герой, фигура, которую не может не заметить английский школьник или даже просто случайный гость Лондона. В обманчиво сокращенном подготовительном курсе школьной истории он возникает на мгновение, одержав бессмысленные победы при Ассайе и Аргауме, потом исчезает, появляется вновь в занимающем половину страницы рассказе о бросках на Торриш-Ведраш и Тулузу, побеждает императора (по всей видимости, в той же кампании) в полном решительности параграфе, отведенном Ватерлоо, а потом выходит на сцену последний раз, позволяя патриотам увенчать его лаврами. Через несколько уроков - в 1828 году - он вдруг выпрыгивает из неизвестности в качестве премьер-министра, а потом временной ``затычки'' на месте Пиля, для того лишь, чтобы с великой помпой и посреди низменной лести его похоронили в 1852-м. В английской истории моего детства Герцог был помечен особенным знаком - знаком доброго персонажа.

Беда Национальных Героев заключается в том, что они представляют собой персоны слишком уж хорошие. Они куда более простоваты и менее занятны, чем откровенные в своей либо черной, либо белой сути герои газетных комиксов, и столь же явно недостоверны. Все эти восковые фигуры великих людей ребенок приемлет молча и не задумываясь (а что ему еще остается делать?), но разум взрослого - если ему случится вникнуть поглубже - с циническим удовлетворением принимается отрицать их. Когда созданные за время долгого мира разоблачительные биографии обрели широкого и заинтересованного читателя, это стало следствием (по моему мнению) не особой испорченности общества, а просто потому, что еще детьми наши сограждане были вынуждены скучать и томиться над длинной серией книг о Национальных Героях, подвизавшихся в политике, флоте, армии, литературе и религии. Эта весьма естественная реакция на чересчур благостное представление прошлого и была использована теми биографами, которые видят смешное во всех - кроме самих себя.

Но Национальный Герой Веллингтон вновь напоминает о себе в названиях многочисленных общественных монументов и улиц. Незнакомый с Лондоном человек, прибывший из Саутгемптона, вне сомнения сойдет с поезда на вокзале Ватерлоо, а такси, весьма возможно, повезет его по Ватерлоо-роад, по мосту Ватерлоо и Веллингтон-стрит к Стрэнду. Если ему случится пройти по Пиккадилли до Гайд-Парк Корнер, он выйдет к ныне темному Эпсли-Хаус, напротив которого находится конная статуя Веллингтона, а за дворцом, уже в самом парке, стоит напыжившийся металлический Ахиллес, отлитый из какой-то пушки, захваченной при Ватерлоо.

Столь настойчивое прославление весьма естественным образом подвигает заатлантического гостя к критике, и я собственными ушами слышал голоса, утверждавшие, и что, в конце концов, никого теперь не волнует Война на Полуострове, и что битву при Ватерлоо, собственно говоря, выиграли пруссаки. Вынужден признать, что было такое время, когда и сам я вносил собственную лепту в непочтительные шепотки и долго относился к Веллингтону (в меру моих тогдашних знаний о нем, безусловно, не слишком глубоких) с неким неуважением. Казалось, он принадлежал к наиболее ненавистному мне типу людей: любитель охоты на лис, аристократ из ирландских протестантов, человек из Дублинского замка, англо-индиец, профессиональный солдат, не умеющий смеяться, а ржущий как жеребец, солдафон, возражавший против отмены телесных наказаний в армии, и самым непростительным образом называвший своих солдат ``подонками и отребьем'', тори, насмерть стоявший против Билля о Реформе 1832 года и противившийся многим, давно назревавшим в Англии переменам, небывалый любитель титулов, орденов и денежных наград.

Сильнейшим источником в стремлении принизить Веллингтона, подействовавшим не на меня одного, явился непрекращающийся поток наполеоновской пропаганды, сумевший перетечь их XIX в XX столетие. Конечно, история - как и прочие установление рода людского - стремится воздать всем и каждому по заслугам, однако есть нечто тревожное в фанатичном раздувании личности этого архидиктатора в эпоху, по крайней мере, хотя бы претендующую на следование принципам демократии. Я несколько раз читал утверждения о том, что о Наполеоне написано больше книг, чем о любом другом человеке. Тому существует, по меньшей мере, две причины; одна заключается в том, что достаточное число человек и, по меньшей мере, одна страна не могут примириться с ликвидацией личной власти и признать равенство людей, существенное для подлинной демократии; другая же кроется в том, что бонапартизм остается активной политической силой во Франции. Таким образом, для культа Наполеона существовали две причины, если не считать присущую всем биографам тенденцию идеализировать своих жертв.

Когда император простился со своими войсками в Фонтенбло, обратившись к ним с речью, откровенно говоря, едва не скрипящей на зубах, он прилюдно объявил о курьезно неадекватном мотиве, побудившем его не искать смерти на поле боя, как, по собственному признанию Наполеона, ему следовало бы поступить.

Он сказал: ``А теперь я должен рассказать на бумаге о том, что мы с вами сделали вместе''.

Утверждение это, возможно, имело чисто риторический характер; оно, наверно, добавляло лишнюю каплю к лживой претензии на обладание ученым и литературным талантом, которой Наполеон с успехом морочил голову всей Франции, или же оно являлось намеком на то дело, которым он и в самом деле занялся после Ватерлоо, - а именно, создание ``наполеоновской легенды'', имеющей своей целью создать династические основания для претензий сына. Наивно предполагать, что отправлявшиеся в Европу с острова Святой Елены манускрипты и записи разговоров имеют какое-либо отношение к исторической правде. Все эти словеса являются пропагандой в пользу короля Римского\footnote{Наполеон II, король Римский (1811-32) - сын и наследник Наполеона I Бонапарта.}; и под это определение попадают также писания маршалов Наполеона, и особенно публикация писем императора, из которых убрано все, что могло бы показать его в недостойном виде или свидетельствовать о какой-нибудь неудаче.

Во многих случаях эта бонапартистская пропаганда могла распространяться преднамеренно и осознанно, особенно между 1815 и 1870 годами, однако ею занимались, не понимая того, и те из более поздних писателей, которые не потрудились ознакомиться с самыми очевидными фактами. Так мистер Эмиль Людвиг заставляет пруссаков отразить последнюю атаку Старой Гвардии (а также дивизий Марконе, Донзело, Башелю) на армию Веллингтона; мистер Хиллари Беллок пишет целую главу о Войне на Полуострове, ни разу не упомянув при этом имени Веллингтона, шестилетнего присутствия в Испании английской армии или операций в Португалии, столь досаждавших маршалам Наполеона и повредивших репутации этих полководцев. Существует много свидетельств того, что до Ватерлоо, когда Веллингтон имел дело только с его маршалами, Наполеон был далек от неуважения к нему. Достаточное количество свидетельств тому может найтись в официальной переписке Наполеона, однако наиболее потрясающий пример содержится в знаменитой беседе с Меттернихом\footnote{Клеменц Меттерних (1773-1859) - министр иностранных дел и фактический глава австрийского правительства в 1809-21, канцлер в 1821-48 годах.}, состоявшейся перед самым возобновлением войны в 1813 году. Император с обычной для него энергией и искусством приводил причины, препятствовавшие вступление Австрии в направленную против него коалицию, и среди наиболее значимых было утверждение о том, что у союзников нет ни одного хорошего генерала; тут Наполеон вдруг поправил себя самого: ``Нет, это не так, у них есть Веллингтон''. Даже в воспоминаниях Наполеона, оставленных во время его нахождения на острове Эльба, я отыскал, по меньшей мере, один пример хорошего и даже уважительного отношения императора к Герцогу; впрочем, после Ватерлоо, подобное отношение по неоднократно упомянутым причинам иссякло.

Таким образом, сторонники Наполеона - как и бонапартисты вообще, - покорно следуя голосу хозяина, без всякого труда обратились к привычке принижать и полностью игнорировать англо-ирландского генерала, который более многих потрудился, чтобы лишить великого человека ореола славы; и вот мы читаем также внушительные и, вероятно, вполне обдуманные ляпусы, какие были сделаны господами Беллоком и Людвигом. Я вынужден признать, что подобно множеству прочих людей был настроен этой пропагандой против Веллингтона лишь потому, что не знал, что имею дело именно с пропагандой, не представляя при этом даже в малейшей степени, где надо искать правильные ответы.

В той мере, насколько это касается меня самого, я имею все основания подозревать, что остался бы бессознательной жертвой блестящей бонапартистской трескотни до конца своей жизни, если бы не чисто случайное событие, которое заставило меня обратиться к чтению книг о Войне на Полуострове и Веллингтоне. Во время моего первого визита на Пиренейский полуостров в начале 1930-х годов я остановился в отеле ``Бусаку'', дворце, строившемся в псевдомавританском стиле для Мануэля II\footnote{Мануэль II (1889-1932) - последний король Португалии. Свергнут в ходе республиканской революции 1910 года и изгнан из страны.}, но оставшемся незаконченным, когда революция заставила короля бежать за границу. Отель расположен в прекрасном саду, точнее парке, полном великолепных деревьев, принадлежавшем монахам старого монастыря. Этот самый монастырь и служил штаб-квартирой Веллингтона перед битвой при Бусаку, где в 18010 году он сражался с маршалом Массена, князем Эсслингом, ``испорченным дитем Победы''. Из путеводителей я узнал, что, по крайней мере, в том случае победа отвернулась от своего испорченного дитяти, потерявшего 5 тысяч человек в серии бесплодных попыток сбросить англо-португальские войска с весьма высокого гребня, прикрывавшего монастырские земли с запада.

На следующий день я поднялся к каменному обелиску, воздвигнутому в память деяний, совершенных португальскими солдатами в битве. Я не военный историк и не кабинетный стратег, но во время первой и безрезультатной мировой войны я получил некоторые наставления в военном деле и провел достаточное количество ценного времени на полях сражений. Рассматривая поле прошлой битвы при Бусаку, с которого давно уже исчезли всяческие следы былого конфликта, я ощутил твердую убежденность в том, что хотя намерение атаковать подобную позицию с востока и можно назвать сущим безумием, но человек, который решил удержать ее и преуспел в своем намерении даже после того, как колонна французских войск поднялась на гребень под прикрытием тумана, далеко не глуп как военачальник. По сути дела, в Бусаку я столкнулся с несколько подчеркнутым образцом ``веллингтоновской позиции'' или результатом того искусного выбора диспозиции, который вкупе с мастерским пониманием возможностей находящихся в его распоряжении войск помогал этому ирландскому охотнику на лис раз за разом побеждать всех искуснейших генералов того времени, включая и их Великого Господина.

Ничто не способно лучше уничтожить предрассудки и указать на невежество, чем предметный урок. Еще не имея более высокого мотива, я начал читать о кампаниях Веллингтона и с течением времени посетил большую часть полей сражений, данных им в Португалии, Испании и на юге Франции. Теперь я желаю о том, что не изучал позиции более внимательно и не делал заметок, но - во всяком случае - я сумел составить собственное приблизительное представление о происходившем, ограничиваясь тем, что посильно обычному человеку. И тогда таинственный ``старый Герцог'', о котором я столько слышал в детстве, начал приобретать в моем взрослом уже сознании облик живого существа, человека, повлиявшего на судьбы мира в большей степени, чем обыкновенно представляется возможным человеку из более или менее заинтересованных кругов в его родной стране. Конечно же, в Англии существует и провеллингтоновская легенда, которая нанесла Герцогу едва ли меньший ущерб, чем антивеллингтоновские козни сторонников Наполеона.

Сильный аристократизм, присущий Веллингтону, и непререкаемо реакционная линия, которой он придерживался как политик, создали в умах людей весьма ошибочное представление о характере Герцога. Не многие люди имели такое количество блестящих врагов при жизни и после смерти. Отталкиваясь от их умного и превратного толкования, как и от неловкого почитания героя, принятого среди его поклонников, не просто составить истинное представление о его характере, особенно при всех апокрифических анекдотах, которых о Веллингтоне рассказано больше, чем о любом другом англичанине. Как ни парадоксально, но по сути своей он был ближе к таким скучноватым и прямолинейным деятелям, как генерал Грант и Вашингтон, чем к тем ярким фигурам, с которыми его обычно ассоциируют. Герцог по доброй воле мирился с мишурой и золотыми кружевами, в которые его столь щедро рядили, однако, следует отметить, что при любой возможности он отбрасывал их в сторону. Герцогу Веллингтону приходилось делать много такого, что не понравилось бы Артуру Уэлсли, как приходилось ладить с людьми, ему не симпатичными; возможно, немногие из знакомых Герцога были ему менее приятны, чем тот тупой полковник, Принц-Регент\footnote{Георг IV (1762-1830) - с 1811 года регент при своем безумном отце Георге III, а затем король Великобритании и Ганновера c января 1820.}, ради которого он добыл столько побед. Веллингтон являлся мастером в области очевидного и неотложного, и наиболее удивительной чертой его интеллекта было некое конское чутье, проницательность, в конечном итоге эквивалентная гениальности, или хотя бы достаточная для того, чтобы победить гения. Крепкий здравый смысл, честность, целостность характера, беспрестанный и усердный труд, решимость добиться повиновения себе и непоколебимая вера в полнейший триумф - вот основные качества, которые позволили ему уцелеть и, в конце концов, победить в долгой и неравной дуэли с колоссальной военной мощью французской империи.

Веллингтон не был наделен литературными или художественными дарованиями и даже вкусом, весьма разумным образом не претендуя на них. Во всяком случае, скромность филистера помогла ему избежать претенциозных ошибок, подчас делавших смешным Наполеона, как случалось, когда он произвел Франциска I из Валуа в Бурбоны, или счел, что в битве при Павии победу одержали итальянцы, а не немцы вместе с испанцами. Тем не менее, Герцог был начитан не хуже Наполеона и, должно быть, много точнее представлял себе те страны, в которых ему приходилось сражаться. К слову сказать, мне нигде не приходилось наталкиваться на примеры ошибок Веллингтона, столь же дурацких, как осмеянное Толстым заблуждение Наполеона, полагавшего, что его будут встречать московские бояре, - спустя целое столетие после того, как и чин их, и сан были упразднены. С другой стороны, Наполеон лучше понимал Францию и французов, нежели Веллингтон Англию и англичан. Тому есть весьма основательная причина: почти до 50-летнего возраста Герцог не был частым гостем в Англии и свои политические принципы усваивал на покоренных территориях, будь то Ирландия или Индия. Любопытна не его принадлежность к тори, а совсем другое: наличие у него такого ума, который позволил ему искренне поверить в то, что даже партия тори должна уступить конституционно выраженной воле народа. Инсинуации французских бонапартистов и некоторых из английских вигов, полагавших, что Веллингтон намеревался стать военным диктатором своей страны, полностью абсурдны. Интересно отметить, что единственным английским генералом, сделавшимся тираном собственной родины, стал не роялист Мальборо и не роялист Веллингтон, но республиканец Кромвель.

Полагаю, что ключ к суровой целостности Веллингтона как общественного деятеля следует искать в словах, процитированных в начале настоящей работы:

\begin{quote}
``Я - ниммуквалла, как мы говорим на Востоке; т.е. я ем соль моего короля, которому считаю своим долгом служить со всем неугасающим рвением и усердием, куда бы король или его правительство ни посчитали бы нужным послать меня''.
\end{quote}

Сентиментальная верность была обычна среди тори, особенно когда их интересы лежали в одной плоскости с рыцарскими устремлениями, но, тем не менее, следует заметить, что Веллингтон говорил о ``короле и его правительстве''; и, оставаясь в рамках того более или менее запутанного переплетения выдумок и аномалий, которое англичане зовут своей конституцией, этот факт позволял Герцогу служить любому правительству, действовавшему от имени короля, даже если оно было враждебно сюзерену. Герцог хотел сказать этим, что он действует как солдат, который является слугой общества, независимым от партийной политики и верным любому законному правительству. Много позже Веллингтон высказал аналогичную мысль, заявив, что считал себя ``защитником и слугой монархии''. Упор на монархию объясняется в этой фразе приливом патриотических чувств, который пожилые джентльмены испытывали в отношении юной королевы Виктории, но в основном тем, что Герцог был практически вынужден сделаться тори, потому что одни только консерваторы верили ему и пользовались его услугами. Конечно, до восшествия на престол королевы Виктории, у него не было причин отдавать свою преданность какому бы то ни было представителю королевской семьи, поскольку, за исключением Георга IV (которого Веллингтон глубоко презирал), все они ненавидели его, и в особенности Йорк, Камберленд, Кларенс (Вильгельм IV) и Георг III. Я отказываюсь поверить в чувство романтической преданности человека, способного сказать о своем сюзерене:

\begin{quote}
``Проклятье! Он так похож словами и внешностью на старину Фальстафа\footnote{Фальстаф - комический персонаж ряда произведений Шекспира. Толстый, добродушный и трусливый пьяница, проводящий время в компании гуляк и распутных девиц, рассказывая им басни о своих воинских подвигах.}, что мне просто стыдно входить в ту комнату, где он находится''.
\end{quote}

Больше шести десятков лет непрерывных трудов отдал обществу Веллингтон, необычайно сознательный и добившийся многих успехов слуга отечества - солдат, администратор, дипломат и (в последние годы) благодетельный хранитель государства. Действительно, он был осыпан наградами, однако у Герцога нередко против его воли вырывался стон.

\begin{quote}
``Отдыха! У любого животного - даже у осла уличного торговца - есть право на отдых, которого нет только у герцога Веллингтона! И выхода нет. Пока я еще в состоянии передвигаться, мне наденут седло на спину и погонят вперед''.
\end{quote}

Быть может, постоянная причина подобной отдачи энергии заключалась в том, что и Корона и Министры не могли отказать себе в возможности использовать практическое, ``конское'' чутье Веллингтона. Понимая, что заменить его будет некем, они пользовались его услугами, пока это было возможно. В 1851 году, когда 82-летний Герцог был уже очень стар, в Хрустальном дворце Гайд-Парка состоялась Первая всемирная выставка. К несчастью, деревья, охваченные огромной конструкцией из стекла и стали, сделались убежищем для множества воробьев, присутствие которых в павильоне совершенно не сочеталось с драгоценными экспонатами, выставленными под их ветвями без всякого покрова. Что же оставалось делать? Стрелять по пернатым из опасения разбить на куски Хрустальный дворец не представлялось возможным, птичий клей действовал медленно и ненадежно. Под натиском общественного мнения Ее Величество послала за герцогом Веллингтоном и, краснея, объяснила суть дела.

\begin{quote}
``Попробуйте использовать ястребов, мадам!'' - услышала она в ответ.
\end{quote}

Жизнь, столь наполненная служением обществу, при всех ее наградах кажется читателю середины XX столетия некой лихорадочной компенсацией за разочарования в интимной сфере. Конечно, Герцог пользовался удивительным успехом, тем не менее, лишь заменявшем отсутствие более прочной связи. Под холодной внешностью и деловитой прямотой таилась куда более эмоциональная натура, чем это принято считать. В официальной переписке Герцога то и дело можно натолкнуться или на письма, полные необыкновенно яда или пессимизма; или вдруг обнаружить едкий и унизительный генеральский приказ, удивляющий и оскорбляющий армию; или же прочитать о необычайно строгом наказании кого-нибудь из подчиненных (скажем, примененном к Рамси\footnote{Джордж Рамси, 9-й граф Далхаузи (1770-1838) - британский генерал и колониальный администратор.} в Витории), совершенно не похожем на спокойное, пусть и несколько негибкое правосудие Герцога, - и в каждом случае это означает, что его нервы на мгновение вырвались из-под контроля нечеловеческой сдержанности. Веллингтон был мужчиной, разочарованным в своих симпатиях и привязанностях. Его добродушный отец скончался, когда Артур еще был мальчишкой, и он рос в тени чрезвычайно одаренного старшего брата, которого обожала мать. Существуют надежные свидетельства того, что леди Морнингтон некоторое время даже презирала Артура, и ему было известно об этом. Тот факт, что она вскоре изменила свое отношение к сыну, никак не мог отменить и само потрясение, и вызванные им последствия. Он никогда более не доверялся женщине, и весьма многочисленные связи не приносили ему удовлетворения. В конце жизни он признавал, что с удовольствием ухаживал за красивыми и привлекательными особами, но при этом едко утверждал, что ни одна из женщин не любила его. Возможно, в этом Герцог ошибался. Похоже, что Веллингтон забыл про собственную жену, однако ее он разлюбил еще до заключения брака, который состоялся лишь благодаря странному институту помолвок, присущему его времени и положению в обществе. Такой человек может найти утешение в привязанности к детям. Но дети Веллингтона были еще младенцами, когда в 1808 году он отправился воевать на полуостров, а в 1818, когда он вернулся домой, они стали для отца юными незнакомцами. Впрочем, нерастраченная любовь его легко и непринужденно разделилась между его племянницами и невестками, старой подругой и ее мужем (Арбутнотами) и детьми других людей.

Победы всегда повергали его в уныние своими неизбежными потерями; утверждают, что Герцог пережил полный душевный надлом, когда генеральный хирург доставил ему первые списки убитых при Ватерлоо. Тем не менее, присланная из Ватерлоо депеша, написанная после этого эмоционального всплеска, являет собой столь сухой и официальный документ, что американский посланник в Лондоне даже не понял, что речь идет о победе, а не о поражении. И это должно послужить предупреждением: трудно, быть может, даже невозможно, вывести истинную природу очень сдержанного и угнетенного человека из огромной массы по большей части официальных документов и из кучи печатных сплетен, в основном, не проверенных в деле.

\hypertarget{ux443ux44dux43bux441ux43bux438---ux430ux43dux433ux43bux438ux447ux430ux43dux435-ux438-ux438ux440ux43bux430ux43dux434ux446ux44b}{%
\chapter{Уэлсли - англичане и ирландцы}\label{ux443ux44dux43bux441ux43bux438---ux430ux43dux433ux43bux438ux447ux430ux43dux435-ux438-ux438ux440ux43bux430ux43dux434ux446ux44b}}

Наша история начинается в 1769 году, который можно считать годом относительного мира, сохраняемого, как казалось, под надежным покровом древних наследственных монархий. Вопреки общепринятому ныне мнению, тогдашние государства вовсе не находились в руках некомпетентных деспотов и услужливых исполнителей их прихотей. Придерживаясь хотя бы капли истины, скажем, Екатерину Великую или Великого Фридриха нельзя назвать неспособными и глупыми правителями. Соотечественники до сих пор видят в Карлосе III Испанском и в португальском маркизе де Помбале\footnote{Себастьян Жозе де Карвалью, 1-й маркиз де Помбал (1699-1782) - дипломат и государственный секретарь, фактически управлявший Португалией в 1750-77 годах и восстановивший Лиссабон после разрушительного землетрясения.} умных и умелых правителей, равных которым в этих странах с тех пор не появлялось. Иосифу II Австрийскому еще предстояло навлечь на себя возмущение подданных, восставших против его демократических реформ, которые консервативный народ просто не смог принять. Швеция ожидала увидеть, как энергичный Густав III положит конец смешным политическим фракциям Шляп и Колпаков\footnote{Две шведские политические фракции XVIII века. ``Колпаки'' выступали за тесные торговые и союзные связи с Россией, ``Шляпы'' - за войну.}, в рамках еще более забавной конституции, определявшей существование четырех законодательных палат, которым полагалось функционировать в мире и гармонии, что они, естественно, сделать попросту не могли. Принятая в Польше оригинальная система выборной монархии показала свою несостоятельность в рожденных ею анархии и гражданских войнах; и могучие соседи прекратили существование такой нелепой власти посредством раздела страны. В малых государствах, даже в могущественной Голландии, царил покой; повсюду в Европе образованные и богатые стремились приобрести тот ``универсальный дух вежливости'', который, в соответствии с мнением мистера Гиббона\footnote{Эдвард Гиббон (1737-94) - британский историк и мемуарист, автор ``Истории упадка и разрушения Римской империи''.}, исходил от французской нации.

Во главе этой величественной, пусть и окостеневшей конструкции, соединенной узами взаимной вражды, находилась Франция - великое и многолюдное королевство. Там же располагалось и самое уязвимое место всей системы, ибо Его Христианнейшее Величество несомненно являлся банкротом, причем пребывал в таком состоянии с 1709 года и поддерживал свое существование лишь благодаря методу, вполне разумно соединявшему секретность с полной неразберихой. Регулярного бюджета не было; фискального состояния не знал сам министр финансов, и хотя он имел возможность публиковать любые цифры, касающиеся доходов и расходов страны, никто не мог помочь ему в этом вопросе. Пусть Людовик XV и не говорил ``после нас хоть потоп'', однако положение дел он понимал лучше многих. Существует вполне достоверный рассказ о том, что, находясь в новой карете вместе с Шуазелем\footnote{Этьен Франсуа де Шуазель (1719-85) - французский премьер-министр и глава дипломатического корпуса в 1758-70 годах.}, король спросил у первого министра, сколько, по его мнению, она стоила. Зная, что короля постоянно обманывают, Шуазель назвал высокую цифру, тем не менее, оказавшуюся достаточно далекой от истины. Возмущенный министр заявил, что настала пора покончить с такого рода поступками. Король едко усмехнулся и предоставил ему полные права на любые действия. После долгого и полного дум молчания Шуазель признал свое бессилие, ибо прекратить и мелкие плутни, связанные с каретой, и крупные жульничества можно было, лишь приступив к разоблачению прочих обманов и злоупотреблений ad infinitum\footnote{``До бесконечности'' (лат.)}; так что заказать новую карету королю уже за разумные деньги можно было только ценой стольких перемен, что они могли бы стоить ему престола. Оба решили не менять положения дел и оставить счет для оплаты потомству.

Конечно, Франция настолько переросла свою прежнюю организацию, что уже в 1769 году эффективная реформа ее ортодоксальными методами была, наверное, невозможна. Во всяком случае, никто не предпринимал таких попыток, и наиболее цивилизованная нация Европы продолжала жить при сложном и расточительном, давно уже устаревшем внутреннем устройстве. Совершенство прозы и урбанические манеры не могли изгладить дефектов всей налоговой системы, и в значительной своей части тогдашние законы и обычаи этой страны можно назвать абсурдными, несправедливыми и вздорными. Но если даже ошибки и преступления ancien regime и в самом деле были политически катастрофичными, важно не забывать, что сама система обладала огромным социальным и культурным очарованием. Часто цитируемая реплика Талейрана о том, что люди, не жившие до 1789 года, никогда не знали подлинной ``la douceur de vivre''\footnote{``Сладости жизни'' (фр.)}, типична среди многих свидетельств. Существовала уникальная цивилизация, и мемуары современников (например, Мармонтеля\footnote{Жан Франсуа Мармонтель (1723-99) - французский писатель, философ и драматург. Наиболее известен в России романом ``Велисарий''.}) свидетельствуют о том, что дары ее ни в коем случае не доставались одной лишь горстке привилегированных лиц, хотя именно они поддерживали и укрепляли ее. Даже суровый и практичный Уильям Питт\footnote{Уильям Питт-старший, 1-й граф Чатем (1708-78) - военный министр в годы Семилетней войны, лидер палаты Общин и премьер-министр от партии вигов в 1766-68.} на свой лад признал это, когда вполне серьезно ответил в Реймсе на вопрос одного аббата:

\begin{quote}
``Действительно, у вас нет политических свобод, однако вы обладаете куда большей гражданской свободой, чем предполагаете сами''.
\end{quote}

Сколь бы мирным ни был тот, 1769 год, но по всей Европе рождались маленькие воины и предводители воинов. И величайший из них родился в августе, на Корсике, как раз во время беспорядков. Некоторое время корсиканцы, возглавляемые героем Босуэлла\footnote{Джеймс Босуэлл (1740-95) - шотландский писатель и мемуарист, друг генерала Корсиканской республики Паоли и автор ``Отчета о Корсике''.} - генералом Паоли\footnote{Паскуали Паоли (1725-1807) - корсиканский политический и военный деятель, глава правительства Корсиканской республики в 1755---69 годах.}, бунтовали против своего сюзерена - одряхлевшей Генуэзской республики. Исполнившись отчаяния, генуэзцы продали Франции свои права, которые реализовать уже не могли, и французы принялись располагать к себе новых подданных проверенным временем методом - затеяв с ними войну. Секретарем корсиканского генерала служил некий происходивший из Флоренции дворянин, Карло Буонапарте. В самый разгар событий его симпатичная жена Летиция родила своего второго сына. Мальчика назвали Наполеоне.

За три месяца до того столь же счастливое событие произошло в семействе англо-ирландской знати: 1 мая 1769 года в Дублине в доме на Мэррион-сквер, леди Энн Уэсли\footnote{В 1799 дети лорда Гаррета и леди Энн изменят родовую фамилию Уэсли (Wesley) на Уэлсли (Wellesley).}, жена Гаррета Уэсли, 1-го графа и 2-го барона Морнингтона из королевства Ирландия, дала жизнь своему шестому ребенку. Забавно отметить, что подлинная фамилия этой недавно возведенной в дворянское достоинство семьи была не Уэсли, а Колли. Английский род Колли, старинный и состоятельный, получил земли в Ирландии от Генриха VIII; однако Уэсли были еще богаче и уверяли, что их предок носил штандарт перед Генрихом II. Потом, уже в XVII столетии, Гаррет Уэсли женился на Элизабет Колли, и поскольку детей у них не было, принялся разыскивать молодого родственника, достойного унаследовать его внушительные поместья. Утверждают, что сие соблазнительное предложение было сделано дальнему английскому родичу, тогда еще посещавшему школу в Вестминстере, Чарльзу Уэсли, брату знаменитого и святого Джона\footnote{Джон Уэсли (1703-91) - английский священнослужитель, богослов и проповедник, основатель методизма.}. Однако предложение (если оно действительно было сделано) получило непредвиденный отказ, и в итоге мистер Гаррет Уэсли усыновил младшего из племянников жены, Ричарда Колли, при условии, что тот примет имя и герб Уэсли.

Поместья, унаследованные Ричардом Колли (отныне Уэсли), располагались в графстве Мит, и обиталищем ему (давно с тех пор уже сгоревшим) служил Данаган, местечко, расположенное недалеко от Дублина. Среди прочих ценностей Ричард унаследовал гнилой городок\footnote{Гнилые местечки - обезлюдевшие в конце XVIII - начале XIX веков деревни и городки в Великобритании, сохранившие при этом представительство в парламенте. Голосами избирателей в этих населенных пунктах часто распоряжался лендлорд, хозяин земли.} Трим, владение которым предоставляло ему право на место в Ирландском парламенте. Этот законодательный орган был в точности скопирован с ``родительского'', заседавшего в Вестминстере, начиная от лорда-канцлера и кончая парламентским приставом, а интересы короля представлял лорд-лейтенант. Однако, на самом деле, сей парламент следовало бы назвать государственным фарсом. Выборы проводились нечасто, лорд-лейтенант обыкновенно отсутствовал, правительственное большинство в обеих палатах обеспечивалось распределением привилегий, синекур и титулов; а до сих пор не отмененный акт Генриха VII запрещал внесение на рассмотрение биллей, не одобренных английским парламентом. Большую часть XVIII столетия управление Ирландией ``осуществлялось'' Примасом\footnote{Примас - почетный титул первого по значению из епископов англиканской и католической церкви.} - с применением силы и подкупа.

Поскольку реальным правителем страны являлся архиепископ, протестантские прелаты образовывали в Ирландии орган куда более значимый и достойный, чем существовавший в Англии. Обыкновенно они были людьми вдохновенными, известными своим патриотизмом и простой, но доходчивой речью. Приятный пример карьеры ирландского экклезиастического патриота представляет собой деятельность доктора Теофилуса Болтона\footnote{Теофилус Болтон (1678-1744) - англиканский епископ, канцлер собора Святого Патрика в Дублине и частый собеседник писателя Джонатана Свифта.}, назначенного в епархию Клопферта; декан Свифт\footnote{Джонатан Свифт (1667-1745) - англо-ирландский сатирик и священник англиканской церкви, декан собора Святого Патрика. Автор ``Путешествий Гулливера''.} немедленно встретился с новым епископом и выразил надежду на то, что его лордство ``послужит своей стране'' в палате Лордов. Епископ возразил:

\begin{quote}
-- Мое епископство очень невелико, и я никогда не получу лучшего, если не услужу двору.\\
-- Когда вы получите лучшее, - заметил Свифт, - то, надеюсь, станете честным человеком. А до той поры - прощайте.
\end{quote}

Встреча и диалог повторились, когда доктор Болтон сделался епископом Эльфина; наконец он достиг поста архиепископа Гашельского, и Свифт вновь явился со своим предложением. На сей раз новоиспеченный архиепископ ответил ему со всей убежденностью:

\begin{quote}
-- Я прекрасно знаю, что ни один ирландец никогда не будет назначен Примасом, и поскольку не могу более увеличить собственное состояние или укрепить положение, то с ревностью буду защищать благосостояние моей страны.
\end{quote}

Это сочетание откровенности с должным вниманием к высшим интересам было распространено среди низшего ирландского протестантского духовенства, - если верить рассказу о преподобном мистере Филлипсе и клойнском епископе. Этот священник с великим гостеприимством принимал у себя своего начальника и подал ему к столу превосходную рыбу. Уже садясь в карету, епископ произнес:

\begin{quote}
-- Мой дорогой Филлипс, вы были чрезвычайно любезны ко мне, но моя благодарность способна возрасти еще больше, если вы утопитесь в речке, которая дает столь отменную рыбу, чтобы я мог передать ваш приход своему сыну Джо.\\
-- Благодарю вашу светлость, - ответил с поклоном священник, - однако я не пожертвую даже последней фалангой самого бесполезного из моих пальцев, чтобы спасти ваше лордство, сына вашей светлости и всех членов вашей семьи от виселицы.
\end{quote}

Многие пытались понять, где и как герцог Веллингтон приобрел эту простую и прямолинейную манеру разговора, столь задевавшую ранимые души; вполне возможно, что он выработал ее, общаясь со своими духовными наставниками и руководителями. Герцог всегда защищал Англиканскую церковь.

\begin{quote}
``Она, - говорил он, - делает нас теми, кто мы есть: нацией честных людей''.
\end{quote}

Некоторое дальнейшее представление об ирландском обществе XVIII столетия вообще и семействе Уэсли в частности могут дать полные сплетен письма миссис Пэндрвес (в девичестве Мэри Гранвилль\footnote{Мэри Гранвилль Делайни (1700-88) - английская художница и мемуаристка. В пожилом возрасте придумала собственную технику создания ботанических иллюстраций, которую называла ``бумажной мозаикой'', и создала около тысячи подобных работ, ныне хранящихся в Британском музее.}), позднее вышедшей вторым браком за ирландского декана, доктора Делайни. Она прибыла в Ирландию из Англии с продолжительным визитом в 1731 году, через три года после того, как мистер Уэсли унаследовал свое состояние. Как подобает, начала она с посещения епископа, доктора Ашера, имевшего в Дублине дом, ``убранный золотого цвета дамаскином, разными ценностями, бюстами и картинами, которые епископ привез с собой из Италии''. Невозможно сказать, что именно представляли собой эти декоративные ``ценности'', но если подобно бюстам и картинам они были привезены из Италии, то от них явно попахивает папизмом.

Миссис Делайни (так ее называть удобнее) была весьма довольна Ирландией или, по крайней мере, той небольшой ее частью, которая, подобно всем состоятельным людям всех времен и народов, жила в основном ради собственного удовольствия и - словно олимпийские боги - находилась выше столь обыкновенных горестей как бедность и труд, стараясь скрыться от них - и притом самым успешным образом - в золотом облаке банкнот. Она с удивлением отметила, что встреченные ею в Ирландии люди в основном ``во всем походили на англичан, являя смесь зла и добра'', и с еще большим удивлением обнаружила, что вели они себя ``очень пристойно'' - в соответствии с собственным рангом. Еще она констатировала наличие в них сердечности и ``великой общительности'', напомнивших ей о Корнуолле и заставивших предположить наличие ``изрядного количества злословия в их обществе''. Неужели это действительно был Дублин?

Вскоре миссис Делайни начала объезжать сельские дома. Еще в жилище епископа Ашера она отметила количество пищи - ``шесть мясных блюд'' за каждой трапезой - и всеобщую приветливость. Но все это померкло в сельской местности, где жители отнюдь не удовлетворялись ``наличием доброго дома, снабженного большим количеством мебели, чем это необходимо''; там она ни разу не присутствовала на обеде, где подавалось бы меньше 14 мясных блюд, или на ужине, где их оказалось бы меньше семи. Очень скоро она и ее друзья отправились с мистером Уэсли на ``увеселительную прогулку'' в Батлерстон, где хрупкая натура сей дамы имела возможность подкрепиться ``холодной дичью, барашком, пирогом с голубями, голландской говядиной, языками, моллюсками, сердцевидками, салатом, изрядным количеством напитков и самым отличнейшим силлабобом (сбитые сливки с вином), который ей приходилось пробовать''. Наевшись до отвала, они отправились музицировать: миссис Делайни - за клавикорды, мистер Уэсли за скрипку, а дети танцевали. ``Мы плясали два часа; потом настала пора ужина, было что выпить и было чем закусить, и мы находились в полном блаженстве''. Компания неохотно рассталась в половине второго ночи.

Быть может, вдохновленная обилием ``выпивки и закуски'' у мистера Уэсли, миссис Делайни приняла приглашение в Данаган, о чем не пожалела. Она сочла, что хозяин дома ``улучшает'' свою собственность деревьями, каналами и классическими статуями. Каналов было три, и на каждом находилась лодка: ``Мы вступаем с музыкой на борт, поднимаем свой флаг и гребем самым гармоническим образом''. Семейство, похоже, в основном обитало в большом зале, где находились орган и клавикорды, и поскольку ``всякий был занят тем, что считал угодным'', обстановка была очень живая, кроме разве что времени семейных молитв. Завтрак, например, начинался в 10 утра и состоял из шоколада, чая, кофе, тостов и масла, а также каудля\footnote{Каудль - горячий и пряный питательный напиток из вина или пива, яиц и пр.}; окончившие трапезу немедленно садились за орган и клавикорды, в то время как остальные играли ракетками в волан. Поскольку в доме много ели, мистер Уэсли настаивал на том, чтобы все проходили не менее четырех миль в день, и обеспечивал гостей ``дорожными посохами''. Не стоит удивляться тому, что миссис Делайни заявила: ``Чем больше я знакома с мистером Уэсли, тем выше мое уважение к нему. Количеством добродетелей и недостатком пороков он превышает всякого известного мне мужчину. Богатство ценно для него лишь тем, что позволяет делать счастливыми окружающих людей''.

Через 17 лет, в 1748 году, миссис Делайни вновь побывала в Данагане. Мистер Уэсли стал теперь лордом Морнингтоном, получив повышение за службу, оставшуюся неведомой для историков, но этот ``приветливый и приятный'' человек по-прежнему был поглощен музыкой, ``улучшениями'' собственных владений, пирами и наделением счастьем окружающих. Появилось и существенное прибавление в семье, ``мой крестник, мистер Уэсли'', который в свои 13 был вундеркиндом. Этот молодой человек, которому предстояло стать отцом герцога Веллингтона и еще четверых удачливых сыновей, вызывает некоторый интерес. Он был ``совершенно расстроен'', когда крестная не позволила ему отсалютовать в честь ее прибытия выстрелами из пушек с его корабля и укрепленного острова. Она добавляет, что это был ``ребенок как ребенок'', ``послушный и управляемый старшими сестрами''.

К сожалению, с этого момента письма миссис Делайни становятся более сдержанными в отношении Уэсли и их интригующего сына. Из других, увы, бесцветных источников мы узнаем только, что молодой мистер Гаррет Уэсли стал депутатом от Трима в 1757 году и очень скоро после этого, в результате кончины отца, сделался лордом Морнингтоном. Миссис Делайни отмечает, что в Дублине открылась музыкальная академия, ``все - леди и джентльмены'', и лорд Морнингтон - ее президент и дирижер. В музыкальной одаренности молодого человека можно не сомневаться. Тринити Колледж в Дублине сделал его доктором музыки, а потом и профессором. Если серьезная музыка его осталась недоступной обыкновенной публике, то песни и гли\footnote{Гли - песня для мужского вокального трио или квартета (в последнем случае высокую партию исполнял мальчик-дискант), без инструментального сопровождения.} - ``С приветом выслушай меня, очаровательная девица'', ``Здесь в прохладном гроте'', ``Приди, прекраснейшая нимфа\ldots{}'' - принесли ему огромную известность. Исполнение некоторых из них сопровождалось овациями и аплодисментами Принца-Регента, которого непочтительный мистер Криви\footnote{Томас Криви (1768-1838) - британский депутат и мемуарист, который 36 лет вел дневник с политическими сплетнями и интервью.} и его приятели называли ``Принни''.

Сам юный лорд тем временем надеялся жениться на дочери герцога, но в самый последний момент объявился соперник, обладающий в два раза большим богатством и в два раза меньшим числом достоинств``. После отказа он обратил внимание на мисс Энн Хилл, дочь бывшего банкира, но племянницу виконта. Быть может, для того, чтобы доказать изменнице, что он, во всяком случае, стоит над низменными материями, лорд Морнингтон отмахнулся от намеков отца невесты, что состояние девицы не соответствует положению жениха. Состояния он не искал; ему было достаточно''вдовьей доли в 1600 фунтов и 500 фунтов на булавки``. Если у невесты и было состояние, то жених был согласен с тем, чтобы его поместили в драгоценные камни. Миссис Делайни покачала своей умудренной мирскими вопросами головкой над подобной романтической сделкой и цинично выразила надежду на то, что''эта особа окажется достойной такого жеста и сделает его счастливым``, многозначительно добавив при этом:''В целом, он очень хороший молодой человек, но кто же без изъяна?"

Встретившись с молодой парой сразу же после медового месяца, она с некоторой неуверенностью отметила, что увидела в них ``пару довольных собой молодых людей, однако ее воспитания недостаточно, чтобы восполнить некоторые пробелы в его образовании''.

Какими бы ни были эти пробелы, очевидно, что лорд Морнингтон проявлял экстравагантность в финансовых вопросах. Он был знаменит не только любовью к музыке, но и ``добротой и гостеприимством''. У него было много детей, из которых выжили восемь. Его сделали графом (по-прежнему ирландским) и это, возможно, кое-что стоило ему. Кроме того, он продолжал начатые отцом ``усовершенствования''. Сад с каналами и классическими статуями превратился в парк площадью 800 акров, с большим озером и скопированной с Версаля прогулочной дорожкой. Посетивший Данаган в 1770-х годах Артур Янг\footnote{Артур Янг (1741-1820) - английский агроном и экономист, автор многочисленных работ по земледелию.} назвал ``посадки'' дорогостоящими и добавил (многозначительно): ``Его лордство занят многочисленными усовершенствованиями''. Озеро увеличилось с 26 до сотни акров, но ``его светлость планирует значительно расширить его''. Всякий, кто в качестве хобби пытался устроить себе загородное имение с ``декоративными усовершенствованиями'', по собственному печальному опыту представляет, что сложно найти более эффективный способ избавиться от денег. Добавим к этому жену-бесприданницу, большое семейство, гостеприимный характер, приобретение титула и ум, скорее обращенный к сочинению ораторий, чем к рентам и земледелию\ldots{}

В 1774 году имение было заложено. Всего за 1000 фунтов, однако настораживает тот факт, что предположительно состоятельный человек сумел добиться лишь такой незначительной суммы. Из-за экономии лорд Морнингтон перевез семью в Лондон, где его преследовал какой-то таинственный долг в 16 000 фунтов и где он портил себе существование всякими сложностями, расходуя на жизнь всего 1800 фунтов в год. Когда его светлость начал покупать лотерейные билеты, стало ясно, что финансовая ситуация очень серьезна. И тут, в мае 1781 года, внезапно и в полном пренебрежении своими обязанностями он умирает в Найтсбридже в возрасте 45 лет, оставив вдову и детей (старшему, Ричарду, едва исполнился 21 год) в ``обстоятельствах, которые, с учетом их общественного положения, следует считать весьма стесненными''. В июле того же года юный граф вынужден увеличить сумму залога до 8000 фунтов, а еще через четыре года - до 10 000. За какую-то половину столетия ``крупное состояние'', над которым столь увлеченно ахала миссис Делайни, полностью расточилось.

\hypertarget{ux441ux435ux43cux435ux439ux43dux44bux439-ux434ux443ux440ux430ux447ux43eux43a}{%
\chapter{Семейный дурачок}\label{ux441ux435ux43cux435ux439ux43dux44bux439-ux434ux443ux440ux430ux447ux43eux43a}}

Такова была жизнь семьи, в которой родился будущий герцог Веллингтон - насколько можно судить по случайно и разрозненной информации. Вполне очевидно, эти ирландские Уэсли были людьми дружелюбными и одаренными, однако, подобно всем членам экономически привилегированного класса, они наслаждались приятной и роскошной жизнью за чужой счет. Музыку, пиры, помпезные катания по каналам, рыцарственные жесты по отношению к мисс Хилл и образование детей - все это оплачивали арендаторы лорда Морнингтона.

Однако о двух первых представителях этой фамилии можно сказать, что они не являлись господами, вечно находящимися в отъезде, и что к жителям своих земель, похоже, относились получше, чем другие помещики к прочим ирландским арендаторам. Единственное доступное нам и полученное из первых рук свидетельство этого оставлено специалистом - агрономом Артуром Янгом, проехавшим в 1776 году по владениям лорда Морнингтона и оставившим нам следующие заметки. ``Хижжины'' (так значится в его собственном написании) имели глинобитные стены толщиной около двух футов, крыши их не протекали, в них было теплее, чем в обыкновенном английском домике, хотя не в каждой из них имелась труба. У всех жителей в изобилии водилась картошка, ``в каждом доме кишела домашняя птица'', в большинстве домов выкармливали свиней и держали, по крайней мере, одну корову. Янг добавляет: ``Если ирландские хижжины останутся такими, какими я их до сих пор видел, то я без всяких колебаний засвидетельствую, что их обитатели живут не хуже, чем большинство английских обывателей''.

Все это очень мило, но в конце своей книги Янг, суммируя впечатления об Ирландии весьма теплыми словами, выражает негодование по поводу угнетения этой страны.

\begin{quote}
``Быть может, закон записан на языке свободы, - утверждает он, - однако бедность может говорить лишь на одном языке - языке рабства''.
\end{quote}

Религиозные и политические притеснения были дополнены экономическими мерами, погубившими ирландскую торговлю, и их единодушно настроенный английский парламент принял под воздействием шумных требований просвещенных промышленников. Поскольку меры эти были в равной мере губительны и для протестантов, и для католиков, для англо-ирландцев и для самих ирландцев, неудивительно, что вся страна бурлила гневом, ненавистью и дышала восстанием.

Возможно, упадок состояния Уэсли в такой же мере объясняется общим экономическим спадом, как и экстравагантными увлечениями графа; однако отъезд в Лондон, конечно же, нельзя объяснить стремлением к экономии, поскольку жизнь в этом городе была много дороже, чем в Дублине или в графстве Мит. Правда заключается в том, что, подобно многим разделявшим их положение людям, Уэсли опасались оставаться в Ирландии. Избавившись от Питта-старшего, Георг III вместе со своими друзьями ввязался в мировую войну против американских колоний, Франции, Испании, Голландии и свирепой лиги враждебных нейтралов, в которую входили Россия, Пруссия и все скандинавские страны. Власть над морем оказалась на время утраченной, британские гарнизоны были временно выведены из Ирландии, возникла серьезная опасность французского вторжения. Под напором этих событий были приняты некоторые меры\footnote{Акт о Папистах 1778 года, расширяющий гражданские права католиков.}, принесшие известное облегчение ирландским католикам и торговле; однако ирландцы вполне простительным образом могут считать, что улучшением положения обязаны скорее давлению извне и оружию ирландских добровольцев (протестантов), чем справедливости и добровой воле.

Во время этой тягучей и мучительной войны был такой период, когда казалось, что вся колоссальная конструкция Британской империи вот-вот обрушится, но даже Георг III и его министры не могли полностью погасить энергию и отвагу того народа, которым они управляли. Американские колонии действительно оказались утраченными, но Уоррен Гастингс\footnote{Уоррен Гастинг (1732-1818) - первый английский генерал-губернатор Индии (1772-85), вместе с Робертом Клайвом считается основателем Британской империи, лишившим французов владычества на Индостане.} и Карлтон\footnote{Гай Карлтон, 1-й барон Дорчестер (1724-1808) - британский военачальник и колониальный администратор, генерал-губернатор Британской Северной Америки.} спасли Индию и Канаду, а суровый Эллиотт\footnote{Джордж Аугустус Элиотт, 1-й барон Хитфилд (1717-90) - британский генерал, губернатор Гибралтара в ходе Великой осады 1779-83.} выдержал трехлетнюю осаду Гибралтара, пока, наконец, вернувший империи власть над морями Родни\footnote{Джордж Бриджес Родни, 1-й барон Родни (1719-92) - британский адмирал, добившийся наибольшей славы в битве у острова Всех Святых в 1782, где разгромил французский флот под командование де Грасса.} не пришел к нему на выручку. Тем не менее, мирные договоры 1783 года оказались столь невыгодными для британкой стороны, что весь мир уже полагал, что Англии не оправиться, и Франция окончательно заняла место ведущей мировой силы. Гордонские бунты июня 1780 года\footnote{Гордонские бунты - антикатоличекие бунты в Лондоне 2-7 июня 1780, начавшиеся после того, как лорд Джордж Гордон, глава ``Ассоциации протестантов'', выступил против Акта о Папистах.}, когда почти в течение целой недели толпа жгла и грабила Лондон, когда погибли или получили ранения более 400 человек, вроде бы продемонстрировали, что состояние дел дома столь же скверное, как и за границей. В известной степени так оно и было.

В подобной остановке личных и общественных несчастий прошли детство и юность Артура Уэсли. Похоже, он был одиноким и неловким мальчишкой, часто болел\ldots{} гадкий утенок, сложный отпрыск семейства. Праздный, сонный, застенчивый, он находился в тени старших братьев - Уильяма\footnote{Уильям Уэлсли-Поул, 1-й барон Мэриборо (1763-1845) - секретарь Адмиралтейства Портленда, Главный секретарь Ирландии (1809-12), глава монетного двора (1814-23).}, принявшего фамилию Поул после награждения поместьем, и Ричарда, 2-го графа, блиставшего в Итоне и Оксфорде. Юный Ричард Морнингтон превосходно усваивал классические науки и привлекал внимание государственных деятелей своей латинской поэзией и греческой ученостью. В те дни подобные достижения вне всяких сомнений сулили блестящие перспективы. Ученый епископ писал занимавшемуся в Оксфорде молодому человеку:

\begin{quote}
``Я не в состоянии более наглядно подчеркнуть значение греческого языка, отменное знание которого зачастую приводит на достойные и выгодные места. К тому же этим языком владел наш Господь''.
\end{quote}

У юного Артура Уэсли не было подобных претензий. Из маленькой школы в Триме возле своего ирландского дома он перекочевал в Челси, в ``академию мистера Брауна'', у которой, похоже, кроме дешевизны других достоинств попросту не было. Соученик вспоминает, что однажды посетивший младшего брата Ричард Морнингтон при расставании с самым серьезным выражением на лице наделил того жалким шиллингом, что самым ядовитым образом подчеркивало стесненные обстоятельства, в которых находился молодой граф. Наконец Артура вместе с младшим братом Джеральдом\footnote{Джеральд Валериан Уэлсли (1770-1848) - доктор богословия, ректор университета Челси, капеллан королевской семьи.} (которому было только девять) послали в Итон, где они безрадостно пребывали в самом конце классного списка. Поскольку Джеральд по прошествии времени сделался доктором богословия, он не мог проявлять такой же невосприимчивости к классическому образованию, как Артур, взявший впоследствии себе за правило никогда не употреблять латинских цитат. В Итоне Артур ни с кем не подружился; напротив, он подрался с ``Бобусом'' Смитом (ставшим спустя годы генеральным прокурором Бенгалии), а потом подлизывался к нему - невзирая на все или же потому, что юный Уэсли был очевидным и явным образом не прав. Он очень редко играл в крикет или занимался греблей, но часто плавал в одиночестве или совершал долгие уединенные прогулки. Так родилась поговорка, утверждающая, что битва при Ватерлоо была выиграна на игровых полях Итона.

Во время долгих каникул Артур Уэсли жил у родственников матери, на севере Уэльса, и там самым близким другом его был подручный кузнеца по фамилии Хьюз. Мальчишки совершали вместе длинные походы по горам, а однажды поссорились, дело дошло до кулаков, и в упорном бою победил подручный кузнеца. Однако, как хвастливо в течение многих лет утверждал мистер Хьюз долгими зимними вечерами в деревенском пабе, мистер Уэсли не рассердился, и дружба их только крепла.

Недовольным родным стало ясно, что Артур ничему не научился в Итоне, а обучение там дорого стоит. Глава семьи Ричард уже перекочевал из неизвестности Ирландской палаты Лордов на место возле мистера Питта-младшего\footnote{Уильям Питт-младший (1759-1806) - премьер-министр Британии в 1783-1801, 1804-06.} в палате Общин и, таким образом, был пристроен. Уильям возложил на себя бремя белого человека от боро\footnote{Боро - небольшой район, городок, имеющий право на самоуправление.} Трим, но что же оставалось делать с Артуром? Он уже превратился в нескладного долговязого 15-летнего юнца, застенчивого, с ломающимся голосом, и настолько неинтересного и неловкого, что девицы на приемах предпочитали не замечать его и позволяли провожать себя кому-нибудь другому. Любимым занятием его было пиликать на скрипке, чему он обучился от родного отца, однако семейство на собственном опыте знало, куда заводят подобные привычки.

Старое правило утверждало, что ``семейный дурачок'' должен служить Церкви, однако Артур даже не мог в достаточной для этого мере осилить латынь и греческий. Оставался единственный вариант: получение поста в армии и надежда на медленный рост по службе - с течением времени и по мере вымирания старших начальников. Запрошенный ради проформы по этому поводу Артур неожиданно отказался. Он не хотел быть солдатом. А чего же он, собственно, хотел? Ну, ему казалось, что из него получится финансист. Идея столь же бестактная, сколь и абсурдная. Ни вдовствующая графиня, ни сам граф не желали слышать напоминаний о том, что их аристократическое древо отмечено родством с банкиром\ldots{} ну и о собственных долгах тоже. И родственники с высшим терпением указали на то, что Артур вправе надеяться лишь на годовое пособие в количестве 125 фунтов от заложенных поместий Ричарда, чего едва ли могло хватить на равную конкуренцию с Бэрингами\footnote{Банк братьев Бэрингов - один из старейших банков мира, основан в 1762. Важнейшей из его ранних сделок являлась покупка Луизианы. Технически Наполеон продал эти земли двум семьям банкиров - Бэрингам и Хоупам, у которых США и выкупили Луизиану.} и Ротшильдами.

Леди Морнингтон уладила это дело, приняв решение, характерное для вдов и наделенное той чрезвычайной практичностью, которую пожилые леди нередко обнаруживают в денежных вопросах. В те дни военная репутация, которой обладает нынешняя Германия, принадлежала Франции; посему Артуру предстояло отправиться во французскую военную академию; но прежде чем он мог это сделать с какой бы то ни было выгодой для себя, ему предстояло выучить французский язык. Таким образом, леди Морнингтон не только сумела перевезти себя и Артура (а возможно, и кого-нибудь из младших детей) из дорогого Лондона в дешевый Брюссель, но и устроила так, чтобы расходы на проживание и обучение были поделены на двоих с неким Джеком Армитансом, сыном состоятельного баронета. Должно быть, в обмен на большую часть расходов юный Армитанс получал бесценный дар: престиж и протекцию леди Морнингтон. Джек Армитанс впоследствии выгодно женился и прожил долгую (он дотянул до 92 лет) и полезную жизнь, каждый день катая на четверке коней из Нортхемптона в Барни к прибытию почтовой кареты и обратно. Он запомнил от проведенного в Брюсселе года лишь то, что они с Артуром Уэсли работали очень мало, много времени проводили в веселых компаниях, где Артур всегда играл на скрипке. Через четыре десятка лет на скачках в Донкастере он услышал: ``Черт меня побери, если это не Джек Армитанс'', и был удостоен рукопожатия герцога Веллингтона, заверившего старого приятеля в том, что, оказавшись в 1815 году в Брюсселе, он позаботился о том, чтобы их прежнему наставнику не причинили никаких тревог и ущерба.

В 1785 году Артура Уэлсли отправили в Анжер, в военную академию, над которой начальствовал тогда инженер Пиньяроль. Однако похоже, выпускала она одних лишь офицеров кавалерии, поскольку основным предметом в ней было обучение верховой езде. В Анжере осталась легенда о Веллингтоне, подобная той, какую сочинили о Наполеоне в Бриенне. В самом деле, жизнь его там представляла прежнюю повесть, полную нездоровья, одиночества и столь явного отсутствия интереса к жизни, что, по сообщениям одного из современников, Артур проводил все свое время ``лежа на софе и играя с маленьким белым терьером''.

В этом есть преувеличение, желание поклонников превратить юного герцога Веллингтона в ``Admirable Criсhton\footnote{``Восхитительный Крайтон'' - сатирическая пьеса Д.М.Барри (создателя ``Питера Пэна'') 1902 года о группе чиновников, попавших на необитаемый остров.}'' лени и невежества. Повесть, конечно, интересна, но как тогда быть с тем, что через год или два мы узнаем от современников о том, ``что он очень много читает'', о его ``огромном интересе к новым изобретениям и открытиям'', о его ``способности к точным и быстрым вычислениям''.

Не столь уж важно, получил Артур Уэсли формальное военное образование в академии Анжера или нет. Главное в том, что первый год или два относительной свободы от ограничений, в период наибольшей впечатлительности в жизни очень молодого человека, он провел в дружеском общении с французской знатью и притом как раз перед революцией. Эти люди сформировали его манеры и вкус, повлияли на стандарты мышления, направили его ум, а также политическое и социальное сознание. Веллингтон - в отличие от Нельсона, Блюхера и пруссаков - отнюдь не ненавидел французов. Он сражался с ними в смертельно опасной дуэли. И сражался, насколько это возможно, в соответствии с неписаными законами чести и любезности, характерными для старого режима.

Анжер лежит в той части Франции, которая в пору революции оставалась католической и настроенной в пользу короля; окружавшие этот город области рождали ужасные контрреволюционные восстания, едва не сокрушившие Республику. Знать, с которой водил знакомство Артур Уэсли, существенно отличалась от богатых версальских придворных, которых, наряду с современными им испанскими грандами, впору считать прикованными к престолу. Многие из этих провинциальных дворян были настолько бедны, что доход их не превышал 50 луидоров в год, однако из их числа вышло большинство офицеров армии, флота и колониальной службы, многие из низших чиновников и провинциальных магистратов. Дочери их становились монашенками или выходили замуж за отставных офицеров, предпочитавших добродетель приданому. Они редко превосходили в ранге полковника или почтмейстера, поскольку высшие, более выгодные посты с одобрения Версаля предоставлялись grande noblesse\footnote{``Великая знать'' (фр.).}. Провинциальная знать, пополнявшая собой вооруженные силы и принимавшая на свои плечи основную тяжесть сражений, бывала в Версале лишь два раза в жизни: в первый, когда полные надежд молодые люди получали возможность поклониться Его Величеству после поступления на службу, и второй, когда горстка выживших и поседевших ветеранов удостаивалась королевского кивка и креста Святого Людовика в награду за пожизненный труд. Невзирая на свою бедность и скудное вознаграждение за труды, они хранили свою традицию - верности, веры и экономной элегантности, и именно о них, а не о разобщенных парижанах думал Берк\footnote{Эдмунд Берк (1729-97) - англо-ирландский политик и один из наиболее влиятельных писателей эпохи Просвещения. Родоначальник идеологии консерватизма. Автор ``Размышления о Французской революции'', утверждавший, что свобода должна оставаться в рамках закона, а реформы - следовать по эволюционному пути.}, превознося ``неподкупное благородство жизни, дешевую защиту народов, пестунью многих сентиментальных и героических предприятий''.

Данаган, Дублин, Кенсингтон, Итон, Брюссель, Анжер - трудно представить себе комбинацию, в большей степени способную утвердить юного аристократа в предрассудках собственной касты. Новые идеи, которым предстояло сокрушить старый режим и править Европой до возрождения безжалостного военного деспотизма нашего времени, еще не проникали в эти края, а в особенности в Анжер. И если Артур Уэсли сталкивался в Анжере с какими-нибудь ограничениями сурового кодекса чести и военной службы, это были требования галантности, греха, которому сочувствовала даже Церковь, - ведь Паоло и Франческа\footnote{Паоло Малатеста и Франческа да Римини - герои ``Божественной комедии'' Данте Алигьери.} пребывают в высшем, а не низшем из кругов ада. В отношениях с католиками Веллингтон никогда не проявлял заметного протестантского хамства - и дома, в Ирландии, и за границей. Но он не выказывал им и доли того благоволения, потока слезливых сантиментов, столь обычных для просвещенных высших сословий Франции перед революцией и через год-другой после ее начала. Гильотину ставили педантичные сентименталисты, проливавшие слезу ``над трогательно дремлющим на солнце добродетельным старцем''.

Если Артур Уэсли положил в основание собственной морали то, что видел в Анжере и вокруг него, он был вполне прав в этом; местные дворяне представляли собой самых лучших из знакомых ему людей; намного превосходящих угнетавших Ирландию обжор и охотников за выгодными местами, великих лондонских тори и вигов с их бесконечной борьбой за кабинет и синекуры и сложными семейными изменами, едва ли смягчавшими причиненное ими зло утратой великодушия. Повторяя в уме великолепные, пусть и направленные не по адресу тирады Берка, Артур Уэсли едва ли мог видеть в своих бывших анжерских друзьях что-либо иное, кроме невинной и героической жертвы рассвирепевшей и обезумевшей революционной толпы, в чем, возможно, и коренится причина его огромной неприязни к народным сборищам и демонстрациям. Некоторые из французских друзей Артура погибли на гильотине, другие пали, защищая собственные дома, третьи были убиты в нантских каменоломнях. Один из них, герцог Бриссак\footnote{Луи де Коссе-Бриссак, 8-й герцог Бриссак (1734-92) - губернатор Парижа в 1771-91, в 1791-92 главнокомандующий Конституционной гвардии, погибший во время Сентябрьских убийств.}, пал в Версале в 1792 году как глава конституционной гвардии Людовика XVI.

Веллингтону повезло в том, что заодно он не заразился французской sensiblerie\footnote{``Преувеличение, деланная сентиментальность'' (фр.)}, которая успела оказать мимолетное влияние даже на столь великолепного человека как Уордсворт\footnote{Уильям Уордсворт (1770-1850) - британский поэт-лирик.}. В последующие годы женщина, не испытывавшая к Веллингтону особенно дружеских чувств, описывала его как ``оригинала, гордого, простого и великого''. И этот человек потенциально уже существовал в загорел 17-летнем юноше, вернувшемся в Англию из Анжера после двухгодичного отсутствия. Шансов проявить оригинальность и величие у него не было и вовсе, простота, должно быть, имелась и - возможно - ее принимали за глупость, гордость же могла находить проявление в сдержанности и неловкости. Тем не менее, подобно всем одаренным людям, он наверняка уже проявлял самые первые и еще неявные признаки латентной энергии и силы. Но увы, насколько нам известно, он не сумел найти себе более взрослого друга, познакомиться с человеком постарше, понимавшим его и выказывавшим симпатию.

Конечно, ни понимания, ни симпатии в собственном доме искать не приходилось - если только он надеялся на это. Леди Морнингтон, дочь банкира, любила материальный успех или хотя бы возможность его, в особенности после того, как жизнь ее прошла с мужем, джентльменом и музыкантом. Оба старших сына ее устроились благополучно\ldots{} что там, они заслуживали поощрения. Вот Уильям, своими достоинствам выслуживший поместье и удивительно преуспевающий в Ирландском парламенте, поддерживая ошибки правительства в качестве представителя трудовой семьи боро Трим. Еще более впечатлял Ричард, глава семейства. Еще 20-летним он с гордостью доказал, что великолепное классическое образование ``дает достойное и доходное место'', ибо он находился в тесной дружбе с мистером Питтом и сидел на скамье казначейства в качестве младшего лорда этого учреждения. Однако Артуром она не гордилась\ldots{} тогда не гордилась. Отсылая сына в Анжер, мать пренебрежительно бросила, что тот ``годен лишь на то, чтобы быть пушечным мясом''.

Знал ли Артур о подобном отношении к нему со стороны своей родительницы? Кто может это сказать\ldots{} Но мать, способная произнести такие слова либо во внезапном порыве материнского раздражения, либо выносив такое суждение, едва ли может показаться сыну любящей и привязанной к нему, и отвергнутый Артур отступил и ушел в себя. Братец Ричард, похоже, разделял отношение матери к этому презренному типу, собственному брату. Он был чересчур занят парламентскими делами, чтобы уделять Артуру какое-то время; и не следует предполагать, что ему было под силу пронзить эту молчаливую отстраненность - при всем увлечении успехами собственного цицероновского красноречия в палате и честью сидеть плечом к плечу с неким непонятным историческим персонажем, носившем прозвище ``трепач Браун, лорд Казначей емкостью в три бутылки''. Однако кое-что Ричард для своего брата сделал. Он написал ему рекомендательное письмо к лорду-лейтенанту Ирландии, где словами, полными достоинства, ощущения собственного превосходства и снисхождения, изложил следующую просьбу.

\begin{quote}
``У меня есть младший брат, - писал он, - которого вы столь любезно обещали иметь в виду относительно назначения в армию. Сейчас он находится здесь и пребывает в полнейшей праздности. Мне безразлично, какую он получит должность, если только назначение состоится скоро''.
\end{quote}

Так, силой братского энтузиазма, Артур был произведен в чин энсина 73-го пехотного полка, которому предстояло отправиться в Индию, в день 7 марта 1787 года.

\end{document}
