% Options for packages loaded elsewhere
\PassOptionsToPackage{unicode}{hyperref}
\PassOptionsToPackage{hyphens}{url}
%
\documentclass[
  oneside,
  12pt,
  titlepage]{book}
\usepackage{lmodern}
\usepackage{amssymb,amsmath}
\usepackage{ifxetex,ifluatex}
\ifnum 0\ifxetex 1\fi\ifluatex 1\fi=0 % if pdftex
  \usepackage[T1]{fontenc}
  \usepackage[utf8]{inputenc}
  \usepackage{textcomp} % provide euro and other symbols
\else % if luatex or xetex
  \usepackage{unicode-math}
  \defaultfontfeatures{Scale=MatchLowercase}
  \defaultfontfeatures[\rmfamily]{Ligatures=TeX,Scale=1}
\fi
% Use upquote if available, for straight quotes in verbatim environments
\IfFileExists{upquote.sty}{\usepackage{upquote}}{}
\IfFileExists{microtype.sty}{% use microtype if available
  \usepackage[]{microtype}
  \UseMicrotypeSet[protrusion]{basicmath} % disable protrusion for tt fonts
}{}
\makeatletter
\@ifundefined{KOMAClassName}{% if non-KOMA class
  \IfFileExists{parskip.sty}{%
    \usepackage{parskip}
  }{% else
    \setlength{\parindent}{0pt}
    \setlength{\parskip}{6pt plus 2pt minus 1pt}}
}{% if KOMA class
  \KOMAoptions{parskip=half}}
\makeatother
\usepackage{xcolor}
\IfFileExists{xurl.sty}{\usepackage{xurl}}{} % add URL line breaks if available
\IfFileExists{bookmark.sty}{\usepackage{bookmark}}{\usepackage{hyperref}}
\hypersetup{
  pdftitle={Wellington's dispatches V},
  hidelinks,
  pdfcreator={LaTeX via pandoc}}
\urlstyle{same} % disable monospaced font for URLs
\usepackage{longtable,booktabs}
% Correct order of tables after \paragraph or \subparagraph
\usepackage{etoolbox}
\makeatletter
\patchcmd\longtable{\par}{\if@noskipsec\mbox{}\fi\par}{}{}
\makeatother
% Allow footnotes in longtable head/foot
\IfFileExists{footnotehyper.sty}{\usepackage{footnotehyper}}{\usepackage{footnote}}
\makesavenoteenv{longtable}
\usepackage{graphicx,grffile}
\makeatletter
\def\maxwidth{\ifdim\Gin@nat@width>\linewidth\linewidth\else\Gin@nat@width\fi}
\def\maxheight{\ifdim\Gin@nat@height>\textheight\textheight\else\Gin@nat@height\fi}
\makeatother
% Scale images if necessary, so that they will not overflow the page
% margins by default, and it is still possible to overwrite the defaults
% using explicit options in \includegraphics[width, height, ...]{}
\setkeys{Gin}{width=\maxwidth,height=\maxheight,keepaspectratio}
% Set default figure placement to htbp
\makeatletter
\def\fps@figure{htbp}
\makeatother
\setlength{\emergencystretch}{3em} % prevent overfull lines
\providecommand{\tightlist}{%
  \setlength{\itemsep}{0pt}\setlength{\parskip}{0pt}}
\setcounter{secnumdepth}{5}
\usepackage[a4paper, margin=22mm]{geometry}
\setlength{\headheight}{15.1pt}
\setlength {\parindent}{2em} 
\usepackage{parskip}
\renewcommand{\baselinestretch}{1.3}
\usepackage{titlesec}
\usepackage{fancyhdr}
\renewcommand{\thechapter}{\Roman{chapter}}
\flushbottom
\pagestyle{fancy} 
\fancyhf{} 
\fancyfoot[CO]{\thepage} 
\fancyhead [RO]{\leftmark} 
\titleformat{\chapter}[display]{\center\Large}{\thechapter}{0.1em}{}
\titlespacing*{\chapter}{0pt}{-50pt}{25pt}
\usepackage{indentfirst}
\usepackage[T2A]{fontenc}
\usepackage[utf8]{inputenc}
\usepackage[russianb]{babel}

\title{Wellington's dispatches V}
\author{}
\date{\vspace{-2.5em}}

\begin{document}
\maketitle

{
\setcounter{tocdepth}{1}
\tableofcontents
}
\hypertarget{part-9-ux434ux435ux43aux430ux431ux440ux44f-1809-ux431ux430ux434ux430ux445ux43eux441}{%
\part*{9 декабря 1809, Бадахос}\label{part-9-ux434ux435ux43aux430ux431ux440ux44f-1809-ux431ux430ux434ux430ux445ux43eux441}}
\addcontentsline{toc}{part}{9 декабря 1809, Бадахос}

\hypertarget{ux43cux435ux43cux43eux440ux430ux43dux434ux443ux43c-ux43eux431-ux43eux43fux435ux440ux430ux446ux438ux44fux445-1809-ux433ux43eux434ux430}{%
\chapter{Меморандум об операциях 1809 года}\label{ux43cux435ux43cux43eux440ux430ux43dux434ux443ux43c-ux43eux431-ux43eux43fux435ux440ux430ux446ux438ux44fux445-1809-ux433ux43eux434ux430}}

Британская армия, предназначенная для службы в Португалии и Испании, была сформирована в конце апреля, за исключением одной не прибывшей пехотной бригады; также некоторые войска оставались в Гибралтаре, ожидая, когда их сменят другие, направленные туда из Португалии\footnote{Веллингтону было разрешено отправить из Лиссабона в гарнизон Гибралатара 30-й и 88-й полки и взамен забрать оттуда 48-й и 61-й.}. Сэр Артур Уэлсли высадился в Лиссабоне 23 апреля.

К этому времени французы завладели Сарагосой, а маршал Сульт удерживал Порту и северные провинции Португалии. 29 марта произошла битва при Медейине\footnote{На самом деле сражение произошло 28 марта 1809; маршал Виктор разгромил испанского генерала Куэсту и оккупировал большую часть Эстремадуры.}; генерал Куэста пытался устранить ее последствия и снова собрать армию в Монастерио, в горах Сьерра-Морены. Французы под командованием маршала Виктора оккупировали Гвадиану, и их передовые позиции находились в непосредственной близости от Лос-Сантоса. Себастиани стоял в Сьюдад-Реале и контролировал армию Ла-Каролины примерно в 12 000 солдат, которой в то время командовал генерал Венегас. Ней завладел Галисией; небольшой французский отряд удерживал Саламанку; Сен-Сир со своим 25-тысячным корпусом был в Каталонии; Келлерманн, сменивший Бессьера на посту командира 6-го корпуса, - в Вальядолиде; а Мортье\footnote{(прим. Гурвуда) Я наоборот считаю, что Мортье ушел из Сарагосы; однако прошло некоторое время, прежде чем он появился в Старой Кастилии.} со своим корпусом и герцог д'Абрантес с 8-м - в Сарагосе.

Португальские войска были на грани уничтожения, в полном беспорядке; а испанские вряд ли имели возможность удержать свои позиции в Сьерра-Морене.

Маркиз де ла Романа, который с января, с момента отплытия британской армии из Коруньи, и до марта находился со своим корпусом неподалеку от Шавеша на границе Португалии, ушел оттуда вместе с армией в сторону Астурии и остановился в этой провинции, когда Сульт через Шавеш вторгся в Португалию.

27 апреля генерал Джон Крэдок передал командование сэру Артуру Уэлсли; и в тот же самый день были отданы приказы о сборе и выступлении армии для подготовки к атаке на Сульта в Порту.

12 мая Сульт был изгнан из Порту, а 18 мая он, преследуемый британскими и португальскими войсками, вошел в Галисию, потеряв все свои пушки, военную казну, боекомплекты, обозы и т.д., и т.д.; он довел из своего корпуса не более 8 000 солдат\footnote{(прим. Гурвуда) См. отчет к военному министру (Каслри) за 18 мая.}.

В это время в других регионах произошли следующие события. Ней, соединившись с Келлерманом, вторгся в Астурию в начале мая и оккупировал ее; маркиз де ла Романа сбежал из Хихона на военном шлюпе.

Жители Виго при помощи капитана Макинли c «Lively» вернули себе этот город; и, воспользовавшись отсутствием Нея и Сульта в Галисии, выдвинули свои отряды к городу Луго, который в итоге атаковали.

Маршал Виктор переправился через Гвадиану 12 или 13 мая и 14-го выделил дивизию, чтобы она пересекла Тежу у Алькантары. Эта дивизия снова отступила обратно несколько дней спустя, вероятно, как только она узнала о случившемся 12-го успехе против Сульта; но сэр Артур Уэлсли, 18 мая прекратив дальнейшее преследование Сульта и 19-го получив сведения о переправе этой дивизии через Тежу, немедленно отдал приказ войскам возвращаться на юг и выступил вместе с ними\footnote{(прим. Гурвуда) См. отчет к военному министру за 20 мая.}.

Передовые войска прибыли в Коимбру, на Мондегу, 26 мая; но сэр Артур Уэлсли, услышав тем временем, что французская дивизия, переправившаяся через Тежу у Алькантары, вернулась обратно за реку, отменил первоначальный приказ о стремительном марше, и британские войска прибыли на Тежу лишь между 7 и 12 июня\footnote{(прим. Гурвуда) См. отчет к военному министру за 31 мая.}.

Они остановились здесь до 27 июня, отчасти для того, чтобы получить поставку денег, недостающей обуви и другого снаряжения, отчасти - чтобы дать отдых людям и лошадям после быстрых маршей, которые они совершили к границе Галисии и обратно.

Также было желательно получить подкрепления в виде 48-го и 61-го полков, ожидаемых из Гибралтара, и 23-х легких драгун, прибывших из Англии, до начала каких-либо дальнейших операций.

За это время французы переместили корпус Нея из Астурии в Галисию; и 6 июня он, вместе с отрядом из корпуса Сульта, предпринял попытку завладеть Виго. Атака с целью захватить мост Сан-Пайо полностью провалилась; равно как и Сульт потерпел неудачу во всех своих попытках привлечь к активным действиям на реке Силл корпус маркиза де ла Романы, который в начале июня снова собрался возле Оренсе на границе Португалии.

Маркиз де ла Романа удалился из Оренсе в сторону Виго, и Сульт решил увести все свои войска из Галисии, оставив корпус Нея в этой провинции в одиночестве\footnote{(прим. Гурвуда) См. письмо Сульта к королю Жозефу, перехваченное у генерала Франчески.}; в конце июня он выдвинулся в Самору на Дуэро, чтобы перевооружить и восстановить свою армию. Ней, обнаружив, что он слишком слаб, чтобы в одиночестве удерживать Галисию, также эвакуировал войска из этой провинции в середине июля и разместился в Асторге\footnote{(прим. Гурвуда) После прочтения этих подробностей и письма Сульта к королю Жозефу, может ли кто-нибудь усомниться в том, что эвакуация Галисии была вызвана операциями британских войск в Португалии? Сульт удалился из Португалии в таком состоянии, что ничего не мог сделать, пока не перевооружился, для чего он отправился в Самору, оставив Нея в Галисии и приказав ему удерживать провинцию. Ней оказывается слишком слабым, чтобы удерживать свои позиции, и отправляется в Асторгу, а провинция остается во владении маркиза де ла Романы (который был слишком слаб, чтобы сражаться с корпусом Сульта в одиночку, даже в его столь ослабленном и искалеченном состоянии), британских моряков и прочих людей в гарнизоне Виго.}.

Как только Виктор обнаружил, что британская армия подошла к Тежу, он начал отступать из Эстремадуры 14 и 15 июня; и, наконец, переправился через Тежу в Альмарасе 24 июня, за два дня до того, как британская армия выступила из Абрантеша и т.д., чтобы двинуться в Пласенсию\footnote{(прим. Гурвуда) Один из ``воплей'' в адрес сэра А. У. основывался на том, что тот задержал свой марш из Абрантеша на столь поздний срок; что, как заявлялось, было вызвано некоторыми спорами с генералом Куэстой по поводу командования или какого-либо другого пустякового вопроса. Во время остановки в Абрантеше через лейтенант-полковника Бурка была организована беседа с генералом Куэстой, но не о командовании или о другом пустяковом вопросе, а о будущих операциях двух армий. Вся корреспонденция попадала к мистеру Фриэ и, вероятно, сейчас находится в канцелярии государственного секретаря по иностранным делам. Но я считаю, что эта остановка на две недели на Тежу не покажется слишком долгой ни одному человеку, который знает, что такое армия, что такое люди и животные, ее составляющие, - что это люди и животные, которые, подобно другим, нуждаются в отдыхе от страшной усталости в жарком климате; что армейские экипажи и прочее снаряжение требуют ремонта и замены после долгого путешествия по плохим дорогам, подобного тому, которое было проделано из Тежу в Галисию и обратно чуть более чем за месяц. Кроме того, армия не могла выступать без денег. Поставка прибыла в Абрантеш 25-го числа; и как только ее получили, был отдан приказ и 27-го начался марш.\\
  Но суть этого ``вопля'' заключается в том, что в результате этой задержки на Тежу Виктор сбежал из Эстремадуры. Пусть кто-нибудь проследит за действиями Виктора с середины мая до конца июня, когда тот, наконец, пересек Тежу, и он увидит, что французы руководствовались полученными ими разведданными о британской армии. Виктор послал корпус переправляться через Тежу и сам пересек Гвадиану, когда обнаружил, что британская армия направилась в Порту, чтобы атаковать Сульта; он вернул этот корпус за реку, когда услышал о результатах атаки на Сульта в Порту; он начал отступать за Тежу 14 и 15 июня, когда узнал, что британская армия прибыла на Тежу в Абрантеш; и он вывел свои войска из Мериды, завершил эвакуацию южной Эстремадуры и пересек Тежу 24 июня, когда обнаружил, что британская армия собирается выступить в сторону Пласенсии. Если все эти факты верны, а все они зафиксированы в официальных отчетах того времени, что изменилось бы в ситуации с Виктором, если бы сэр А. У. выступил из Абрантеша на несколько дней раньше? Только то, что Виктор удалился бы за Тежу на несколько дней раньше!!!}.

\end{document}
