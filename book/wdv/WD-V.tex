% Options for packages loaded elsewhere
\PassOptionsToPackage{unicode}{hyperref}
\PassOptionsToPackage{hyphens}{url}
%
\documentclass[
  oneside,
  12pt,
  titlepage]{book}
\usepackage{lmodern}
\usepackage{amssymb,amsmath}
\usepackage{ifxetex,ifluatex}
\ifnum 0\ifxetex 1\fi\ifluatex 1\fi=0 % if pdftex
  \usepackage[T1]{fontenc}
  \usepackage[utf8]{inputenc}
  \usepackage{textcomp} % provide euro and other symbols
\else % if luatex or xetex
  \usepackage{unicode-math}
  \defaultfontfeatures{Scale=MatchLowercase}
  \defaultfontfeatures[\rmfamily]{Ligatures=TeX,Scale=1}
\fi
% Use upquote if available, for straight quotes in verbatim environments
\IfFileExists{upquote.sty}{\usepackage{upquote}}{}
\IfFileExists{microtype.sty}{% use microtype if available
  \usepackage[]{microtype}
  \UseMicrotypeSet[protrusion]{basicmath} % disable protrusion for tt fonts
}{}
\makeatletter
\@ifundefined{KOMAClassName}{% if non-KOMA class
  \IfFileExists{parskip.sty}{%
    \usepackage{parskip}
  }{% else
    \setlength{\parindent}{0pt}
    \setlength{\parskip}{6pt plus 2pt minus 1pt}}
}{% if KOMA class
  \KOMAoptions{parskip=half}}
\makeatother
\usepackage{xcolor}
\IfFileExists{xurl.sty}{\usepackage{xurl}}{} % add URL line breaks if available
\IfFileExists{bookmark.sty}{\usepackage{bookmark}}{\usepackage{hyperref}}
\hypersetup{
  pdftitle={Wellington's dispatches V},
  hidelinks,
  pdfcreator={LaTeX via pandoc}}
\urlstyle{same} % disable monospaced font for URLs
\usepackage{longtable,booktabs}
% Correct order of tables after \paragraph or \subparagraph
\usepackage{etoolbox}
\makeatletter
\patchcmd\longtable{\par}{\if@noskipsec\mbox{}\fi\par}{}{}
\makeatother
% Allow footnotes in longtable head/foot
\IfFileExists{footnotehyper.sty}{\usepackage{footnotehyper}}{\usepackage{footnote}}
\makesavenoteenv{longtable}
\usepackage{graphicx,grffile}
\makeatletter
\def\maxwidth{\ifdim\Gin@nat@width>\linewidth\linewidth\else\Gin@nat@width\fi}
\def\maxheight{\ifdim\Gin@nat@height>\textheight\textheight\else\Gin@nat@height\fi}
\makeatother
% Scale images if necessary, so that they will not overflow the page
% margins by default, and it is still possible to overwrite the defaults
% using explicit options in \includegraphics[width, height, ...]{}
\setkeys{Gin}{width=\maxwidth,height=\maxheight,keepaspectratio}
% Set default figure placement to htbp
\makeatletter
\def\fps@figure{htbp}
\makeatother
\setlength{\emergencystretch}{3em} % prevent overfull lines
\providecommand{\tightlist}{%
  \setlength{\itemsep}{0pt}\setlength{\parskip}{0pt}}
\setcounter{secnumdepth}{5}
\usepackage[a4paper, margin=22mm]{geometry}
\setlength{\headheight}{15.1pt}
\setlength {\parindent}{2em} 
\usepackage{parskip}
\renewcommand{\baselinestretch}{1.3}
\usepackage{titlesec}
\usepackage{fancyhdr}
\renewcommand{\thechapter}{\Roman{chapter}}
\flushbottom
\pagestyle{fancy} 
\fancyhf{} 
\fancyfoot[CO]{\thepage} 
\fancyhead [RO]{\leftmark} 
\titleformat{\chapter}[display]{\center\Large}{\thechapter}{0.1em}{}
\titlespacing*{\chapter}{0pt}{-50pt}{25pt}
\usepackage{indentfirst}
\usepackage[T2A]{fontenc}
\usepackage[utf8]{inputenc}
\usepackage[russianb]{babel}

\title{Wellington's dispatches V}
\author{}
\date{\vspace{-2.5em}}

\begin{document}
\maketitle

{
\setcounter{tocdepth}{1}
\tableofcontents
}
\hypertarget{part-8-ux430ux432ux433ux443ux441ux442ux430-1809-ux434ux435ux43bux435ux439ux442ux43eux441ux430}{%
\part*{8 августа 1809, Делейтоса}\label{part-8-ux430ux432ux433ux443ux441ux442ux430-1809-ux434ux435ux43bux435ux439ux442ux43eux441ux430}}
\addcontentsline{toc}{part}{8 августа 1809, Делейтоса}

\hypertarget{ux431ux440ux438ux433ux430ux434ux43dux43eux43cux443-ux433ux435ux43dux435ux440ux430ux43bux443-ux440.ux43aux440ux43eux444ux443ux440ux434ux443}{%
\chapter{Бригадному генералу Р.Крофурду}\label{ux431ux440ux438ux433ux430ux434ux43dux43eux43cux443-ux433ux435ux43dux435ux440ux430ux43bux443-ux440.ux43aux440ux43eux444ux443ux440ux434ux443}}

Мой дорогой генерал,

Я планировал встретиться с вами сегодня, но здесь необходимо уладить столько проблем, что я вынужден отложить свой визит на завтра.

Меры, которые должны быть приняты на вашей позиции на Тежу, так сильно зависят от нашего общего положения, что нет никакой возможности приступить к их воплощению без разъяснения всей нашей ситуации, чем я и собираюсь вас побеспокоить.

Судя по тому, что я узнал, противник уже привел или в данный момент ведет через Пуэрто-де-Баньос в Пласенсию все свои войска из Кастилии. В письме от 4-го числа из Альмейды генерал Бересфорд сообщает мне, что сейчас никого из них нет в Кастилии, и что ушедшие насчитывают 34 000 человек. Это доводит французские силы в этой части Испании примерно до 70 000 человек.

Либо они насядут на нас этим корпусом на правом фланге и пробьются в Кордобу; либо снова попробуют сыграть в ложные маневры и направят большой корпус в Португалию, чтобы отделить нас от испанцев.

Не думаю, что они попытаются форсировать Тежу. В самом деле, успех где угодно выше Альмараса не принесет им особой пользы, поскольку они не смогут провести свою армию никаким путем, кроме того, по которому мы пришли, и на котором можно было бы легко обороняться.

Вот то, чего мы должны остерегаться: во-первых, это марш в Португалию; во-вторых, попытка пробиться к Кордобе сквозь правый фланг испанцев.

Могло бы оказаться полезным на будущее, если бы мы восстановили мост в Альмарасе, но в данный момент об этом не может быть и речи. Даже если нам предложат заманчивую возможность нанести удар, нам потребуются отдых и еда для людей и лошадей, прежде чем мы сможем воспользоваться этим шансом.

Я считаю, что касательно марша в Португалию или защиты переправы через Тежу у Альмараса следует принять следующие меры.

Во-первых, мы должны разрушить мост через эту реку на правом берегу, сохранив доски, и перенести их на левый берег.

Во-вторых, мы должны убрать лодки, привязанные к этой стороне моста. Лодки можно увести к высохшему притоку чуть ниже переправы, а балки и доски отправить к холму за переправой.

В-третьих, мы должны провести строительные работы на тех позициях, которые лучше всего позволят нам защитить переправу.

Приняв эти меры, мы получим возможность защищать брод, если противник попытается его форсировать; с другой стороны, если мы обнаружим, что противник движется в Португалию, и мы будем вынуждены идти в том направлении, это место возьмет под свою охрану испанская дивизия из Меса-де-Ибор, а мы сможем сжечь материалы для постройки мостов; или, если окажется целесообразным перейти реку, мы сможем легко эти мосты восстановить.

Надеюсь, что завтра я пришлю вам несколько пушек. Сегодня днем я отправлю к вам инженера и инструменты для строительства сооружений, которые могут вам потребоваться.

Я предполагал, что на батареях у моста располагалась испанская артиллерия; но я был дезинформирован и по этому поводу, и по прочим вопросам. Я написал генералу Куэсте, высказав пожелание, чтоб испанская артиллерия крупного калибра могла быть послана для защиты переправы в Альмарасе.

Я приказал отправить вам провизию. Надеюсь, что с этого дня мы будем получать наши стандартные пайки.

Доверьтесь мне,

Артур Уэлсли

\hypertarget{ux434ux43eux43dux443-ux43cux430ux440ux442ux438ux43dux443-ux434ux435-ux433ux430ux440ux430ux44e}{%
\chapter[Дону Мартину де Гараю]{\texorpdfstring{Дону Мартину де Гараю\footnote{Мартин де Гарай-и-Пералес (1771-1822) - испанский экономист, генеральный секретарь и депутат Верховной Хунты, автор избирательной процедуры в Кадисские кортесы; после реставрации Фердинанда - министр финансов, инициатор либеральной налоговой реформы.}}{Дону Мартину де Гараю}}\label{ux434ux43eux43dux443-ux43cux430ux440ux442ux438ux43dux443-ux434ux435-ux433ux430ux440ux430ux44e}}

Сэр,

Я имел честь получить письмо, которым Ваше Превосходительство удостоили меня 31 июля, выразив в нем одобрение Центральной хунты касательно поведения британской армии под моим командованием в ходе боевых действий 29 июля.

Я очень хорошо понимаю всю ценность одобрения Центральной хунты и прошу вас передать ей мою почтительную признательность.

Особенно мне льстит доверие, которое она оказала мне, назначив меня одним из генерал-капитанов испанской армии; и сегодня я написал генеральному секретарю Его Величества, чтобы попросить представить Его Величеству свидетельство одобрения и доверия Центральной хунты; а также ходатайствовать, чтобы мне разрешили принять назначение в испанскую армию, которым правительство столь любезно меня удостоило.

Пока не будет получен ответ Его Величества, я буду счастлив оказать правительству все услуги, которые будут в моих силах.

Честь имею,

Артур Уэлсли

\hypertarget{ux434ux43eux43dux443-ux43cux430ux440ux442ux438ux43dux443-ux434ux435-ux433ux430ux440ux430ux44e-1}{%
\chapter{Дону Мартину де Гараю}\label{ux434ux43eux43dux443-ux43cux430ux440ux442ux438ux43dux443-ux434ux435-ux433ux430ux440ux430ux44e-1}}

Сэр,

В отдельном письме я выразил признательность правительству за оказанную мне честь о назначении генерал-капитаном испанской армии, и теперь я должен поблагодарить их за лошадей, которых они были столь любезны подарить мне от имени Его Величества короля Фердинанда VII.

Что касается жалования за звание генерал-капитана, я надеюсь, что правительство простит меня, если я откажусь обременять финансы Испании во время этого сражения за ее независимость.

Честь имею,

Артур Уэлсли

\hypertarget{ux432ux438ux43aux43eux43dux442ux443-ux43aux430ux441ux43bux440ux438-ux433ux43eux441ux443ux434ux430ux440ux441ux442ux432ux435ux43dux43dux43eux43cux443-ux441ux435ux43aux440ux435ux442ux430ux440ux44e}{%
\chapter{Виконту Каслри, государственному секретарю}\label{ux432ux438ux43aux43eux43dux442ux443-ux43aux430ux441ux43bux440ux438-ux433ux43eux441ux443ux434ux430ux440ux441ux442ux432ux435ux43dux43dux43eux43cux443-ux441ux435ux43aux440ux435ux442ux430ux440ux44e}}

Милорд,

1-го числа я известил Вашу Светлость о выступлении французского корпуса к Пуэрто-де-Баньос и о возможных помехах для операций нашей армии, которые может вызвать его прибытие в Пласенсию; с тех пор появилось так много затруднений, что они заставили нас отступить и занять оборонительную позицию на Тежу, и я вынужден побеспокоить вас подробностями о произошедшем.

Когда я вошел в Испанию, я поддерживал контакт с генералом Куэстой через сэра Роберта Уилсона\footnote{Роберт Уилсон (1777-1849) - британский генерал, участник кампаний во Фландрии, Египте, Пруссии (при австрийском штабе) и на Полуострове; комиссар при штабе Кутузова в 1812 и в Заграничных походах; автор нескольких исторических работ, в том числе знаменитых дневников о событиях в России} и полковника Роуча\footnote{Филипп Китин Роуч (1783-1829) - британский офицер на испанской службе, связной Веллингтона с союзными генералами; участник битвы при Кастайе} по поводу оккупации Пуэрто-де-Баньос и Пуэрто-де-Пералес; по первому пункту, в конце концов, было принято решение, что его должен занять корпус, сформированный под командованием маркиза де ла Рейны - в его состав должны были войти два батальона армии генерала Куэсты и два батальона из Бехара; а о Пуэрто-де-Пералес следовало позаботиться герцогу дель Парке\footnote{Висенте де Каньяс-и-Портокарреро, герцог дель Парке (1751-1824) - испанский генерал, участник войны Конвента и Войны за независимость; присягнул Жозефу Бонапарту, но сменил сторону после Байлена; командовал испанской армией при Тамамесе и Альба-де-Тормес; был послом в России в 1801-02; входил в состав революционных Кортесов в Либеральное трехлетие} с отрядами из гарнизона Сьюдад-Родриго. Я сомневался, что гарнизон Сьюдад-Родриго сможет выделить туда отряд; но почти не беспокоился касательно первого вопроса, так что в письме от 17 июля я попросил маршала Бересфорда взглянуть на Пералес и написал, что считаю Баньос защищенным, как видно из выдержки, которую я прилагаю.

30-го в Талавере были получены сведения, что на 28-е в Фуэнте-Робле заказаны 12 000 пайков, а в Лос-Сантосе, на тот же день, - 24 000 для французского корпуса, который, как предполагалось, шел к Пуэнто-де-Баньос. Генерал Куэста выразил некоторую обеспокоенность по поводу этой позиции и послал мне сообщение, в котором предложил отправить туда корпус сэра Роберта Уилсона. В тот день сэр Роберт был в Талавере, но его корпус находился в горах около Эскалоны; и поскольку он приносил пользу в этом районе и располагался около Мадрида, обеспечивая связь, которую я хотел поддерживать, я предложил, чтобы в Баньос был немедленно отправлен испанский корпус. Я не смог убедить генерала Куэсту, хотя он определенно признавал необходимость подкрепления, предлагая послать в Баньос сэра Роберта Уилсона; в той же степени он понимал, какую пользу принесет делу отправка сэра Роберта обратно в Эскалону.

В то время у нас не было никаких дополнительных сведений о наступлении врага, кроме тех заказанных пайков; и у меня была надежда, что противник будет вынужден отказаться от продвижения благодаря разведданным о нашем успехе 28-го числа; и что войска в Пуэрто смогут защитить себя; и я считал, что в этих обстоятельствах было бы нежелательно уводить сэра Роберта из Эскалоны.

Однако 31-го я вновь, в письме генералу О'Донохью\footnote{Томас де О'Донохью-и-О'Риан (1764-\ldots) - испанский генерал ирландского происхождения; адъютант и переводчик при штабе Куэсты; генеральный инспектор пехоты}, копию которого прилагаю, стал просить генерала Куэсту отправить туда достаточно сильную испанскую дивизию, но это не принесло результата; он отказывался выделить генерала Бассекура\footnote{Луи Алехандро де Бассекур-и-Дюпир (1769-1826) - испанский генерал французского происхождения; участник войны Конвента под командованием генерала Рикардоса; сражался при Медейине, Туделе и Уклесе, был назначен генерал-капитаном Валенсии и Куэнки} до утра 2-го, когда мы узнали, что враг вошел в Бехар; и стало очевидно, что войска в Пуэрто не выстоят.

2-го числа мы получили сообщение о том, что две колонны противника вошли в Пласенсию. Маркиз де ла Рейна, два батальона которого насчитывали всего лишь 600 человек и по 20 патронов на каждого солдата, отступил из Пуэрто и Пласенсии, не сделав ни единого выстрела; и ушел к мосту Альмараса, который, как он заявил, намеревался разрушить. Батальоны Бехара разбежались, не оказав сопротивления.

В тот день меня посетил генерал Куэста и предложил половине армии выдвинуться в тыл, чтобы противостоять врагу, а другой половине сохранить свою позицию в Талавере. Я сказал ему: если под половиной армии он имеет в виду половину каждой из наших армий, я отвечу лишь одно - что я готов уйти или остаться со всей британской армией, но не могу разделить ее. Затем он попросил меня выбрать - уйду я или останусь; и я предпочел уйти, полагая, что британские войска, скорее всего, будут эффективными при отсутствии споров; а также потому, что обеспечение коммуникаций через Пласенсию было для нас более важной задачей, чем для испанской армии; хотя и для них очень важной. Этим решением, похоже, генерал Куэста был полностью удовлетворен.

Наступление врага на нашем фронте, начавшееся 1-го числа, заставило меня думать, что, отчаявшись выбить нас из Талаверы, они намеревались прорваться через Эскалону и, таким образом, открыть сообщение с французским корпусом, идущим из Пласенсии.

Это подозрение подтвердилось ночью 2-го числа в письмах, полученных от сэра Роберта Уилсона, копии которых я прилагаю; и прежде чем я покинул Талаверу 3-го, я прислушивался к мнению генерала О'Донохью, беседовал с ним касательно всей нашей ситуации и указывал ему на возможность того, что в случае прорыва врага через Эскалону генерал Куэста будет вынужден сам покинуть Талаверу, прежде чем я смогу вернуться к нему; и я призвал его собрать все телеги, которые только можно достать, чтобы вывезти наш госпиталь. Он пожелал, чтобы я изложил суть этого разговора в письменной форме и послал ему письмо, копию которого я прилагаю, для передачи генералу Куэсте.

3-го числа британская армия выдвинулась к Оропесе, а испанский корпус генерала Бассекура - к Сентинейо; где я хотел, чтобы он остановился на следующий день, чтобы я мог держаться поближе к нему.

Около пяти часов вечера я услышал, что французы прибыли из Пласенсии в Навальмораль, и поэтому они оказались между нами и мостом Альмараса.

Примерно через час я получил письмо от генерала О'Донохью и дополнения к нему, копии которых прилагаю; в них сообщалось о намерении генерала Куэсты выйти вечером из Талаверы и оставить там мой госпиталь, за исключением тех людей, которых можно переместить имеющимися у него средствами, потому что он опасался, что я недостаточно силен для вражеского корпуса, идущего из Пласенсии; и потому что противник обошел его фланг, вернувшись в Санта-Олайю прямо напротив линии фронта.

Я признаю, что эти причины не показались мне достаточными, чтобы отказаться от столь важной позиции, как Талавера; чтобы подвергнуть объединенные армии одновременной атаке и с фронта, и в тыл; и чтобы оставить там мой госпиталь; и я написал ему письмо, копию которого прилагаю.

К сожалению, это письмо дошло до генерала после того, как он выступил; и он прибыл в Оропесу вскоре после рассвета утром 4-го числа.

Тогда нам следовало решить, что делать. Противник, численность которого, как утверждают, составляет 30 000 человек; который в любом случае включает в себя корпуса Сульта и Нея, объединенные или не очень удаленные друг от друга; и который, по предположению Жозефа Бонапарта и маршала Журдана, достаточно силен, чтобы атаковать британскую армию в 25 000 человек, - с одной стороны, владел главной дорогой, ведущей к переправе через Тежу в Альмарасе, где, как мы знали, мост был разрушен, хотя некоторые лодки все еще оставались на реке. С другой стороны, у нас были причины ожидать наступления корпуса Виктора на Талаверу, как только станет известно о марше генерала Куэсты; и после того, как он оставил 12 000 человек для наблюдения за Венегасом и понес 10-11 000 потерь убитыми и ранеными к концу боя, его корпус должен был составлять 25 000 человек. Мы могли выбраться из этой сложной ситуации лишь быстро передвигаясь (в чем наши войска уступали французам, т.к. несколько дней не получали припасы), или благодаря успехам в двум сражениях: если в каком-нибудь из них мы бы потерпели поражение, то лишились бы возможности отступить; и если бы Сульт и Ней, избегая боя, оставили нас и дождались прибытия Виктора, то мы бы подверглись совместной атаке 50 000 человек без шансов на отступление. У нас были причины ожидать, что Сульт уничтожит лодки в Альмарасе, так как маркиз де ла Рейна не смог увести их от реки.

Нашим единственным путем к отступлению был мост Арсобиспо; и если бы мы продвинулись куда-то еще, противник, разрушив мост, в то время как армия была бы вынуждена вступить в бой с Сультом и Неем, лишил бы нас этой единственной возможности.

Мы не могли занять позицию в Оропесе, так как тем самым оставили бы открытой дорогу из Талаверы к мосту Арсобиспо через Калеру; и после тщательного изучения всей ситуации я пришел к выводу, что было бы целесообразно отступить к мосту Арсобиспо и занять оборонительную позицию на Тежу.

Я был вынужден принять этот последний вариант, потому что французы теперь насчитывают по меньшей мере 50 000 человек, способных противостоять объединенным армиям, и корпус из 12 000 человек для слежения за Венегасом; и я также придерживался мнения, что чем раньше будет занята оборонительная позиция, тем большей будет вероятность, что войска смогут ее защитить.

В результате я выступил 4-го и перешел Тежу по мосту Арсобиспо; и продолжил свой путь к этой позиции, на которой удачно расположился для защиты переправы в Альмарасе и низовья Тежу. Генерал Куэста перешел реку в ночь на 5-е и все еще находится на мосту Арсобиспо.

Около 2000 раненых вывезены из Талаверы, еще 1500 остались там; и я сомневаюсь, что при каких-либо обстоятельствах было бы возможным или гуманным попытаться вытащить оттуда еще кого-нибудь из них. Судя по обращению с некоторыми солдатами, ранеными 27-го числа и попавшими в руки врага, и по тому, как я всегда заботился об их раненых, которые оказались у меня, я могу ожидать, что с этими людьми будут хорошо обращаться; и мне остается лишь сожалеть о том, что недавнее стечение обстоятельств, над которыми, исходя из сложившейся ситуации, я не имел и не мог иметь никакого контроля, поставило армию в это положение, когда она вынуждена оставить кого-то из своих людей позади.

Честь имею,

Артур Уэлсли

\hypertarget{ux432ux438ux43aux43eux43dux442ux443-ux43aux430ux441ux43bux440ux438-ux433ux43eux441ux443ux434ux430ux440ux441ux442ux432ux435ux43dux43dux43eux43cux443-ux441ux435ux43aux440ux435ux442ux430ux440ux44e-1}{%
\chapter{Виконту Каслри, государственному секретарю}\label{ux432ux438ux43aux43eux43dux442ux443-ux43aux430ux441ux43bux440ux438-ux433ux43eux441ux443ux434ux430ux440ux441ux442ux432ux435ux43dux43dux43eux43cux443-ux441ux435ux43aux440ux435ux442ux430ux440ux44e-1}}

Мой дорогой Лорд,

Я мало что могу добавить к моему публичному письму за эту дату, которое, я надеюсь, оправдает меня, избавив от всякой вины в глазах министров Его Величества, за исключением той, что я иногда доверял испанскому генералу. Мы были бы в безопасности, если бы я мог убедить его занять Баньос, как и следовало поступить; и мы могли бы избежать позора потери госпиталя, если бы он отослал генерала Бассекура ночью 30-го или утром 31-го, или если бы он сохранил свои позиции в Талавере.

На самом деле, я действительно верю, что спас обе армии целиком, решив отступить к Арсобиспо и перейдя рубеж Тежу, как только обнаружил врага в Навальморале и узнал, что генерал Куэста безвозвратно покинул Талаверу.

Мы не победим в Эстремадуре все войска маршалов - Сульта, Нея, Мортье, Виктора - Келлермана, Себастиани, а также короля и еще 5000 человек Сюше.

Бересфорд писал мне 4-го числа из Алмейды, что 34 000 человек ушли через Баньос в Пласенсию и что в Кастилии не осталось никого, кроме больных. Я рекомендовал Хунте направить Роману\footnote{Педро Каро-и-Суредо, маркиз де ла Романа (1761-1811) - испанский капитан Армады, участник захвата Менорки и осады Гибралтара; командир касадоров в войне Конвента, генерал, попавший в плен к французам в Дании в 1808, и предводитель герийеров в Испании после своего побега}, герцога дель Парке и герийеров к Мадриду.

Что теперь будут делать французы, когда они собрали все свои силы? Либо они атакуют правый фланг испанской армии под командованием Венегаса и начнут наступление на Кордобу; либо они попробуют другой обходной маневр и вторгнутся в Португалию через Каштелу-Бранку; либо они попытаются форсировать Тежу в Альмарасе, по единственной доступной им переправе.

В любом из этих вариантов у нас есть возможность отступления; и в случае вторжения в Португалию через Каштелу-Бранку я приказал Бересфорду передвинуть его корпус вправо в сторону Каштелу-Бранку, а сам пойду в Португалию и пересеку Тежу у Абрантиша.

Я прошу вас сделать с моим донесением и приложениями к нему то, что вы сочтете наилучшим: опубликуете их полностью, или только часть, сохранив в тайне приложения; или составите выписку из донесения. В любом случае я прошу вас показать всю депешу и приложения моим братьям Поулу\footnote{Уильям Уэлсли-Поул, 3-й граф Морнингтон, 1-й барон Мэриборо (1763-1845) - британский политик, секретарь Адмиралтейства Портленда; главный секретарь Ирландии (1809-12), глава монетного двора (1814-23)} и Генри\footnote{Генри Уэлсли, 1-й барон Коули (1773-1847) - британский дипломат, секретарь посольства в Стокгольме (1792), личный секретарь Ричарда Уэлсли в Индии (1797-99, 1801-02), посол в Испании (1810-22), в Австрии (1823-31), во Франции (1835, 1841-46)}, а копию отправить герцогу Ричмонду\footnote{Чарльз Леннокс, 4-й герцог Ричмонд (1764-1819) - британский политик, лорд-лейтенант Ирландии в 1807-13, генерал-губернатор Британской Северной Америки в 1818-19}.

Я прошу вас ознакомиться с описанием французских операций, которое сделал Журдан для Сульта; и посмотреть, насколько точно они были информированы обо всех наших передвижениях и намерениях. Обратите особое внимание на то, что Виктор прекрасно знал 23-го, что мы должны атаковать его 24-го. Он мог узнать об этом только из разведданных, полученных из нашего лагеря, потому что ни одна из частей наших войск не готовилась к атаке до наступления темноты, когда дивизия Бассекура выдвинулась в Кардиэль; и действительно, Виктор начал свое отступление той ночью, еще до того, как Бассекур вышел из Талаверы; но до сих пор считается, что у французов нет разведки в Испании.

Чарльз\footnote{Чарльз Стюарт Вейн, 3-й маркиз Лондондерри (1778-1854) - британский генерал и политик; генерал-адъютант Веллингтона, комиссар при прусском штабе в 1813, ранен при Кульме; посол в Австрии в 1814-23; автор ``Истории Полуостровной войны''; сводный брат Каслри и издатель его корреспонденции} расскажет вам, как сильно мы страдаем из-за провизии.

Доверьтесь мне,

Артур Уэлсли

\hypertarget{ux434ux43eux441ux442ux43eux43fux43eux447ux442ux435ux43dux43dux43eux43cux443-ux434ux436ux43eux43dux443-ux432ux438ux43bux44cux435ux440ux441ux443}{%
\chapter[Достопочтенному Джону Вильерсу]{\texorpdfstring{Достопочтенному Джону Вильерсу\footnote{Джон Чарльз Вильерс (1757-1838) - британский посол в Португалии в 1808-10}}{Достопочтенному Джону Вильерсу}}\label{ux434ux43eux441ux442ux43eux43fux43eux447ux442ux435ux43dux43dux43eux43cux443-ux434ux436ux43eux43dux443-ux432ux438ux43bux44cux435ux440ux441ux443}}

Мой дорогой Вильерс,

Прилагаю мои публичные и частные письма к лорду Каслри, которые сообщат вам точное положение дел в этом районе.

Сейчас все в порядке, и я бы совсем ни о чем не беспокоился, будь у нас провизия: но мы почти голодаем.

Прилагаю меморандум, который я передал полковнику Мюррею\footnote{Джордж Мюррей (1772-1846) - британский генерал, участник кампании во Фландрии и Полуостровной войны; вице-губернатор Канады в 1814-15; начальник штаба оккупационной армии в 1815-18; военный министр в 1828-30} несколько дней назад, с просьбой привезти товары из Лиссабона. Я буду очень вам признателен, если вы выразите пожелание, чтобы правительство согласовало с генерал-интендантом этапы перевозки этих предметов на телегах из Лиссабона в Бадахос.

Доверьтесь мне,

Артур Уэлсли

\hypertarget{part-9-ux434ux435ux43aux430ux431ux440ux44f-1809-ux431ux430ux434ux430ux445ux43eux441}{%
\part*{9 декабря 1809, Бадахос}\label{part-9-ux434ux435ux43aux430ux431ux440ux44f-1809-ux431ux430ux434ux430ux445ux43eux441}}
\addcontentsline{toc}{part}{9 декабря 1809, Бадахос}

\hypertarget{ux43cux435ux43cux43eux440ux430ux43dux434ux443ux43c-ux43eux431-ux43eux43fux435ux440ux430ux446ux438ux44fux445-1809-ux433ux43eux434ux430}{%
\chapter{Меморандум об операциях 1809 года}\label{ux43cux435ux43cux43eux440ux430ux43dux434ux443ux43c-ux43eux431-ux43eux43fux435ux440ux430ux446ux438ux44fux445-1809-ux433ux43eux434ux430}}

Британская армия, предназначенная для службы в Португалии и Испании, была сформирована в конце апреля, за исключением одной не прибывшей пехотной бригады; также некоторые войска оставались в Гибралтаре, ожидая, когда их сменят другие, направленные туда из Португалии\footnote{Веллингтону было разрешено отправить из Лиссабона в гарнизон Гибралатара 30-й и 88-й полки и взамен забрать оттуда 48-й и 61-й.}. Сэр Артур Уэлсли высадился в Лиссабоне 23 апреля.

К этому времени французы завладели Сарагосой, а маршал Сульт удерживал Порту и северные провинции Португалии. 29 марта произошла битва при Медейине\footnote{На самом деле сражение произошло 28 марта 1809; маршал Виктор разгромил испанского генерала Куэсту и оккупировал большую часть Эстремадуры.}; генерал Куэста пытался устранить ее последствия и снова собрать армию в Монастерио, в горах Сьерра-Морены. Французы под командованием маршала Виктора оккупировали Гвадиану, и их передовые позиции находились в непосредственной близости от Лос-Сантоса. Себастиани стоял в Сьюдад-Реале и контролировал армию Ла-Каролины примерно в 12 000 солдат, которой в то время командовал генерал Венегас. Ней завладел Галисией; небольшой французский отряд удерживал Саламанку; Сен-Сир со своим 25-тысячным корпусом был в Каталонии; Келлерманн, сменивший Бессьера на посту командира 6-го корпуса, - в Вальядолиде; а Мортье\footnote{(прим. Гурвуда) Я наоборот считаю, что Мортье ушел из Сарагосы; однако прошло некоторое время, прежде чем он появился в Старой Кастилии.} со своим корпусом и герцог д'Абрантес с 8-м - в Сарагосе.

Португальские войска были на грани уничтожения, в полном беспорядке; а испанские вряд ли имели возможность удержать свои позиции в Сьерра-Морене.

Маркиз де ла Романа, который с января, с момента отплытия британской армии из Коруньи, и до марта находился со своим корпусом неподалеку от Шавеша на границе Португалии, ушел оттуда вместе с армией в сторону Астурии и остановился в этой провинции, когда Сульт через Шавеш вторгся в Португалию.

27 апреля генерал Джон Крэдок передал командование сэру Артуру Уэлсли; и в тот же самый день были отданы приказы о сборе и выступлении армии для подготовки к атаке на Сульта в Порту.

12 мая Сульт был изгнан из Порту, а 18 мая он, преследуемый британскими и португальскими войсками, вошел в Галисию, потеряв все свои пушки, военную казну, боекомплекты, обозы и т.д., и т.д.; он довел из своего корпуса не более 8 000 солдат\footnote{(прим. Гурвуда) См. отчет к военному министру (Каслри) за 18 мая.}.

В это время в других регионах произошли следующие события. Ней, соединившись с Келлерманом, вторгся в Астурию в начале мая и оккупировал ее; маркиз де ла Романа сбежал из Хихона на военном шлюпе.

Жители Виго при помощи капитана Макинли c «Lively» вернули себе этот город\footnote{27 марта 1809 шотландский капитан фрегата Джордж МакКинли поддержал повстанцев Виго, устроив бомбардировку города с моря и подготовив возможность штурма с суши.}; и, воспользовавшись отсутствием Нея и Сульта в Галисии, выдвинули свои отряды к городу Луго, который в итоге атаковали.

Маршал Виктор переправился через Гвадиану 12 или 13 мая и 14-го выделил дивизию, чтобы она пересекла Тежу у Алькантары. Эта дивизия снова отступила обратно несколько дней спустя, вероятно, как только она узнала о случившемся 12-го успехе против Сульта; но сэр Артур Уэлсли, 18 мая прекратив дальнейшее преследование Сульта и 19-го получив сведения о переправе этой дивизии через Тежу, немедленно отдал приказ войскам возвращаться на юг и выступил вместе с ними\footnote{(прим. Гурвуда) См. отчет к военному министру за 20 мая.}.

Передовые войска прибыли в Коимбру, на Мондегу, 26 мая; но сэр Артур Уэлсли, услышав тем временем, что французская дивизия, переправившаяся через Тежу у Алькантары, вернулась обратно за реку, отменил первоначальный приказ о стремительном марше, и британские войска прибыли на Тежу лишь между 7 и 12 июня\footnote{(прим. Гурвуда) См. отчет к военному министру за 31 мая.}.

Они остановились здесь до 27 июня, отчасти для того, чтобы получить поставку денег, недостающей обуви и другого снаряжения, отчасти - чтобы дать отдых людям и лошадям после быстрых маршей, которые они совершили к границе Галисии и обратно.

Также было желательно получить подкрепления в виде 48-го и 61-го полков, ожидаемых из Гибралтара, и 23-х легких драгун, прибывших из Англии, до начала каких-либо дальнейших операций.

За это время французы переместили корпус Нея из Астурии в Галисию; и 6 июня он, вместе с отрядом из корпуса Сульта, предпринял попытку завладеть Виго. Атака с целью захватить мост Сан-Пайо полностью провалилась\footnote{7-9 июня корпус Нея был разгромлен испанскими войсками полковника Пабло Морийо при попытке французов захватить Виго и Понтеведру.}; равно как и Сульт потерпел неудачу во всех своих попытках привлечь к активным действиям на реке Силл корпус маркиза де ла Романы, который в начале июня снова собрался возле Оренсе на границе Португалии.

Маркиз де ла Романа удалился из Оренсе в сторону Виго, и Сульт решил увести все свои войска из Галисии, оставив корпус Нея в этой провинции в одиночестве\footnote{(прим. Гурвуда) См. письмо Сульта к королю Жозефу, перехваченное у генерала Франчески.}; в конце июня он выдвинулся в Самору на Дуэро, чтобы перевооружить и восстановить свою армию. Ней, обнаружив, что он слишком слаб, чтобы в одиночестве удерживать Галисию, также эвакуировал войска из этой провинции в середине июля и разместился в Асторге\footnote{(прим. Гурвуда) После прочтения этих подробностей и письма Сульта к королю Жозефу, может ли кто-нибудь усомниться в том, что эвакуация Галисии была вызвана операциями британских войск в Португалии? Сульт удалился из Португалии в таком состоянии, что ничего не мог сделать, пока не перевооружился, для чего он отправился в Самору, оставив Нея в Галисии и приказав ему удерживать провинцию. Ней оказывается слишком слабым, чтобы удерживать свои позиции, и отправляется в Асторгу, а провинция остается во владении маркиза де ла Романы (который был слишком слаб, чтобы сражаться с корпусом Сульта в одиночку, даже в его столь ослабленном и искалеченном состоянии), британских моряков и прочих людей в гарнизоне Виго.}.

Как только Виктор обнаружил, что британская армия подошла к Тежу, он начал отступать из Эстремадуры 14 и 15 июня; и, наконец, переправился через Тежу в Альмарасе 24 июня, за два дня до того, как британская армия выступила из Абрантеша и т.д., чтобы двинуться в Пласенсию\footnote{(прим. Гурвуда) Один из ``воплей'' в адрес сэра А. У. основывался на том, что тот задержал свой марш из Абрантеша на столь поздний срок; что, как заявлялось, было вызвано некоторыми спорами с генералом Куэстой по поводу командования или какого-либо другого пустякового вопроса. Во время остановки в Абрантеше через лейтенант-полковника Бурка была организована беседа с генералом Куэстой, но не о командовании или о другом пустяковом вопросе, а о будущих операциях двух армий. Вся корреспонденция попадала к мистеру Фриэ и, вероятно, сейчас находится в канцелярии государственного секретаря по иностранным делам. Но я считаю, что эта остановка на две недели на Тежу не покажется слишком долгой ни одному человеку, который знает, что такое армия, что такое люди и животные, ее составляющие, - что это люди и животные, которые, подобно другим, нуждаются в отдыхе от страшной усталости в жарком климате; что армейские экипажи и прочее снаряжение требуют ремонта и замены после долгого путешествия по плохим дорогам, подобного тому, которое было проделано из Тежу в Галисию и обратно чуть более чем за месяц. Кроме того, армия не могла выступать без денег. Поставка прибыла в Абрантеш 25-го числа; и как только ее получили, был отдан приказ и 27-го начался марш.\\
  Но суть этого ``вопля'' заключается в том, что в результате этой задержки на Тежу Виктор сбежал из Эстремадуры. Пусть кто-нибудь проследит за действиями Виктора с середины мая до конца июня, когда тот, наконец, пересек Тежу, и он увидит, что французы руководствовались полученными ими разведданными о британской армии. Виктор послал корпус переправляться через Тежу и сам пересек Гвадиану, когда обнаружил, что британская армия направилась в Порту, чтобы атаковать Сульта; он вернул этот корпус за реку, когда услышал о результатах атаки на Сульта в Порту; он начал отступать за Тежу 14 и 15 июня, когда узнал, что британская армия прибыла на Тежу в Абрантеш; и он вывел свои войска из Мериды, завершил эвакуацию южной Эстремадуры и пересек Тежу 24 июня, когда обнаружил, что британская армия собирается выступить в сторону Пласенсии. Если все эти факты верны, а все они зафиксированы в официальных отчетах того времени, что изменилось бы в ситуации с Виктором, если бы сэр А. У. выступил из Абрантеша на несколько дней раньше? Только то, что Виктор удалился бы за Тежу на несколько дней раньше!!!}.

Корпус Келлермана эвакуировался из Астурии и Бискайи, вернувшись в Вальядолид; а корпус Мортье находился в Леоне. Сюше, победивший Блейка в Бельчите 18 июня\footnote{Франко-польский корпус Сюше разбил испанские войска генерала Блейка рядом с Сарагосой и вытеснил его из Арагона; в ходе битвы произошел подрыв пороха на правом фланге испанцев, унесший жизнии более 200 человек.}, ушел обратно в Сарагосу; корпус Сен-Сира был занят осадой Жироны.

Испанская армия под командованием генерала Куэсты, усиленная кавалерией и пехотой и с необычайной быстротой перевооруженная после сражения при Медейине, выдвинулась из Монастерио в начале мая, когда Виктор перешел Гвадиану для поддержки дивизии, отправленной к мосту в Алькантаре; и генерал Куэста атаковал укрепленный пост, который враг оставил в Мериде. Однако он не преуспел в этом нападении; противник сохранил свой пост в Мериде, и генерал Куэста развернул свою штаб-квартиру в Фуэнте-дель-Маэстре до конца июня, когда противник вывел свои войска из Эстремадуры и пересек Тежу в Альмарасе, узнав о прибытии британской армии в Абрантеш на этой реке.

В конце июня генерал Куэста разместил свою штаб-квартиру в Пуэрто-де-Мирабете, напротив Альмараса, отправив одну из своих дивизий в Арсобиспо.

Таким образом, в конце июня испанская армия под командованием Куэсты была на Тежу; французы Виктора - в Талавере-де-ла-Рейна; Себастьяни ушел из Сьюдад-Реала и находился неподалеку от Толедо; а корпус Венегаса, который также был усилен, продвинулся в Ла-Манчу. Французы эвакуировали Галисию, за исключением корпуса Нея, который покинул эту провинцию позднее и прибыл в Асторгу в середине июля. Мортье находился в Леоне, Сульт - в Саморе, Келлерман - в Вальядолиде, Сюше в Сарагосе, а Сен-Сир занимался блокадой Жироны. Британская армия вышла из Абрантеша и т.д. 27 июня и выступила на Пласенсию, чтобы сотрудничать с испанскими войсками в попытке вытеснить французов с юга Испании.

Командующий королевскими войсками в Португалии единолично несет ответственность за эту операцию, мотивы которой были различными. Во-первых, имея в виду общее состояние войны в Испании, а также в Германии, казалось желательным предпринять некоторые усилия в это время в Испании. Во-вторых, средства были сочтены адекватными относительно поставленной цели. Генерал Куэста имел под своим непосредственным командованием 38 000 человек, а генерал Венегас - 18 000; британская армия насчитывала не менее 20 000 человек, не считая бригады генерала Роберта Кроуфурда, которая высадилась в Лиссабоне 28 июня и должна была немедленно начать свой марш, чтобы присоединиться к армии. Этим войскам должны были противостоять 28 000 человек Виктора и 12 000 Себастиани; и все, кого король мог выставить из гарнизона Мадрида и собственной гвардии.

До начала июля не было известно ни то, что Сульт вывел свои войска из Галисии в крайне непригодном для службы состоянии, как видно из перехваченных писем, полученных от генерала Франчески; ни то, что Ней покинул эту провинцию и ушел в Асторгу до конца июля\footnote{(прим. Гурвуда) Я не думаю, что мы были уверены в эвакуации Нея из Галисии, пока он не вошел в Эстремадуру в августе.}; а Мортье и Келлерман все еще должны были находиться в Астурии и в Бискайе.

Были просчитаны сложности операции; однако предполагалось, что приказы испанского правительства предоставят транспортные средства и провизию, которые были нам необходимы или же могли потребоваться в дальнейшем, поскольку испанцы выражали сильнейшее рвение к сотрудничеству с британскими войсками. Транспортные средства, как известно, в этой стране были, и предполагалось, что собираемый урожай повлечет за собой изобилие провизии.

\end{document}
